% main.tex -- Komplettes Skript in einer Datei
\documentclass[11pt,a4paper]{article}
\usepackage[a4paper,margin=2.2cm]{geometry}
\usepackage[T1]{fontenc}
\usepackage[utf8]{inputenc}
\usepackage[ngerman]{babel}
\usepackage{lmodern} % scalable Latin Modern fonts for microtype/expansion
\usepackage{microtype}
\usepackage{amsmath,amssymb,mathtools}
\usepackage{siunitx}
\usepackage{booktabs}
\usepackage{enumitem}
\usepackage{graphicx}
\usepackage{xcolor}
\usepackage{hyperref}
\usepackage{placeins}
\usepackage{float}

\usepackage{tikz}
\usepackage{circuitikz}
\usepackage{pgfplots}
\usepgfplotslibrary{groupplots}
\pgfplotsset{compat=1.18}
\usetikzlibrary{arrows.meta,positioning,calc,shapes.geometric}
\tikzset{
  block/.style={draw, rectangle, minimum height=9mm, minimum width=16mm, align=center},
  sum/.style={draw, circle, inner sep=0pt, minimum size=4.5mm},
  branch/.style={circle, fill, inner sep=0pt, minimum size=1.6mm},
  line/.style={-Latex, thick},
  thin/.style={-Latex, semithick}
}

\hypersetup{colorlinks=true,linkcolor=blue,urlcolor=blue,citecolor=blue}

% Optionale Zwischendateien gefahrlos einbinden (nur falls vorhanden)
\newcommand{\einbinden}[1]{\IfFileExists{#1}{\input{#1}}{\typeout{Zwischendatei nicht gefunden: #1}}}
\newcommand{\Laplace}{\mathcal{L}}
\newcommand{\p}{p} % Laplace-Variable
\newcommand{\dd}{\mathrm{d}}
\newcommand{\jj}{\mathrm{j}}
\newcommand{\Ree}{\operatorname{Re}}
\newcommand{\Imm}{\operatorname{Im}}
\newcommand{\argg}{\operatorname{arg}}

\title{Skript Regelungstechnik (LaTeX-Version)}
\author{}
\date{\today}

\begin{document}
\maketitle
% kapitel/00-vorbemerkung.tex
\noindent
Dieses Skript ist eine gesetzte (LaTeX-)Fassung der hochgeladenen handschriftlichen Notizen.
Begriffe und Formeln wurden dabei sprachlich gegl\"attet und fachlich konsistent formuliert
(z.\,B.\ eindeutige Definitionen von \"Ubertragungsfunktion, Frequenzgang, Kreisverst\"arkung und
Stabilit\"atskriterien).

\tableofcontents
\newpage

% kapitel/analoge-uebertragungsglieder.tex
\section{Analoge \"Ubertragungsglieder (\"UTG)}
\subsection{Begriffe und Eigenschaften}
Ein \"Ubertragungsglied (\"UTG) beschreibt den Zusammenhang zwischen einem Eingangssignal
(\glqq Ursache\grqq) $x_e(t)$ und einem Ausgangssignal (\glqq Wirkung\grqq) $x_a(t)$:
\[
  x_e(t)\;\longrightarrow\;\text{\"UTG}\;\longrightarrow\;x_a(t).
\]
Im Folgenden betrachten wir in erster Linie analoge, \textbf{lineare} und \textbf{zeitinvariante} Systeme
(\textbf{LZI}-Systeme). F\"ur diese Systeme gelten besonders einfache Rechenregeln
(Superposition, Faltung, Laplace- und Frequenzbereichsdarstellung).

\paragraph{Linearit\"at (Superpositionsprinzip).}
Ein System ist linear, wenn f\"ur beliebige Signale $x_1, x_2$ und Konstanten $a,b$ gilt:
\[
  x_e(t)=a\,x_1(t)+b\,x_2(t)\quad\Rightarrow\quad
  x_a(t)=a\,y_1(t)+b\,y_2(t),
\]
wobei $y_i(t)$ die jeweilige Antwort auf $x_i(t)$ ist.

\paragraph{Zeitinvarianz.}
Ein System ist zeitinvariant, wenn eine zeitliche Verschiebung des Eingangssignals
eine identische Verschiebung des Ausgangssignals bewirkt (bei gleichen Anfangsbedingungen):
\[
  x_e^{\ast}(t)=x_e(t-t_0)\quad\Rightarrow\quad x_a^{\ast}(t)=x_a(t-t_0).
\]

\paragraph{Beispiel (LZI).}
Ein Verst\"arker mit $x_a(t)=2\,x_e(t)$ ist linear und zeitinvariant.
Eine Rampe $x_e(t)=t\,\sigma(t)$ wird zu $x_a(t)=2t\,\sigma(t)$.

% kapitel/typische-signale.tex
\section{Typische Signale}
\subsection{Aperiodische Eingangssignale}
\paragraph{Sprungfunktion (Einheitssprung).}
\[
  \sigma(t)=
  \begin{cases}
    0, & t<0,\\
    1, & t\ge 0.
  \end{cases}
\]
Ein allgemeiner Sprung der H\"ohe $x_{e0}$ lautet $x_e(t)=x_{e0}\,\sigma(t)$.

\paragraph{Rechteckimpuls.}
Ein Rechteckimpuls (H\"ohe $x_{e0}$, Dauer $t_0$) l\"asst sich als Differenz zweier Spr\"unge schreiben:
\[
  x_e(t)=x_{e0}\bigl(\sigma(t)-\sigma(t-t_0)\bigr).
\]

\paragraph{Beispiel (Rechteckimpuls).}
F\"ur $x_{e0}=2$ und $t_0=0{,}5$ gilt
$x_e(t)=2\bigl(\sigma(t)-\sigma(t-0{,}5)\bigr)$.

\paragraph{Rampe.}
Die (Einheits-)Rampe ist das Integral der Sprungfunktion:
\[
  r(t)=\int_{0}^{t}\sigma(\tau)\,\dd\tau = t\,\sigma(t).
\]


% Typical input signals (step, ramp, rectangular pulse)
\begin{figure}[tb]
  \centering
  \begin{tikzpicture}
    \begin{groupplot}[
      group style={group size=2 by 2, horizontal sep=1.4cm, vertical sep=1.0cm},
      width=0.46\textwidth, height=0.28\textwidth,
      grid=both, grid style={line width=.1pt, draw=gray!20},
      major grid style={line width=.2pt, draw=gray!35},
      xlabel={$t$}, xmin=-0.5, xmax=2,
      ymin=-0.1, ymax=2.1,
      axis lines=left,
      tick label style={font=\small},
      label style={font=\small},
      title style={font=\small},
    ]
      \nextgroupplot[title={Sprung $\sigma(t)$}, ylabel={$x_e(t)$}]
        \addplot[thick] table[x=t,y=step,col sep=comma] {data/signals_step_ramp_pulse.csv};
        \addplot[only marks, mark=*, mark size=1.2pt] coordinates {(0,0)}; % discontinuity marker
      \nextgroupplot[title={Rampe $r(t)=t\,\sigma(t)$}]
        \addplot[thick] table[x=t,y=ramp,col sep=comma] {data/signals_step_ramp_pulse.csv};
      \nextgroupplot[title={Rechteckimpuls $\sigma(t)-\sigma(t-t_0)$}]
        \addplot[thick] table[x=t,y=pulse,col sep=comma] {data/signals_step_ramp_pulse.csv};
        \addplot[dashed] coordinates {(1,0) (1,1)};
      \nextgroupplot[title={Diracimpuls $\delta(t)$}]
        \addplot[thick, ->] coordinates {(0,0) (0,1.6)};
        \addplot[only marks, mark=*, mark size=1.2pt] coordinates {(0,0)};
        \node[anchor=west, font=\small] at (axis cs:0.05,1.55) {$\delta(t)$};
    \end{groupplot}
  \end{tikzpicture}
  \caption{Typische Eingangssignale im Zeitbereich (normiert, $t_0=1$) inklusive Diracimpuls.}
  \label{fig:signals_typical}
\end{figure}








\subsection{Impulsfunktion (Dirac-Sto\ss)}
Die Dirac-Impulsfunktion $\delta(t)$ ist eine idealisierte Grenzfunktion; im strengen mathematischen Sinn ist $\delta(t)$ keine (klassische) Funktion, sondern eine Distribution (verallgemeinerte Funktion).
Sie wird als Idealisierung eines sehr kurzen Impulses mit endlichem Fl\"acheninhalt verstanden und besitzt die Eigenschaften
\[
  \int_{-\infty}^{\infty}\delta(t)\,\dd t = 1,
  \qquad
  \delta(t)=0\ \text{f\"ur}\ t\neq 0.
\]
Wichtige Beziehung zur Sprungfunktion:
\[
  \frac{\dd}{\dd t}\sigma(t)=\delta(t).
\]

\subsection{Antwortfunktionen eines LZI-\"UTG}
\begin{itemize}[itemsep=2pt]
  \item \textbf{\"Ubergangsfunktion (Sprungantwort)} $h(t)$: Antwort auf den Einheitssprung
  $x_e(t)=\sigma(t)$, also $x_a(t)=h(t)$.
  \item \textbf{Gewichtsfunktion (Impulsantwort)} $g(t)$: Antwort auf den Dirac-Impuls
  $x_e(t)=\delta(t)$, also $x_a(t)=g(t)$.
\end{itemize}
Wegen $\delta(t)=\frac{\dd}{\dd t}\sigma(t)$ gilt f\"ur LZI-Systeme der zentrale Zusammenhang:
\[
  g(t)=\frac{\dd}{\dd t}h(t)=\dot h(t).
\]
(Umgekehrt: $h(t)=\int_0^t g(\tau)\,\dd\tau$.)



% kapitel/zeitbereich-beschreibung.tex
\section{Mathematische Beschreibung im Zeitbereich}
\subsection{Lineare Differentialgleichung mit konstanten Koeffizienten}
Ein LZI-\"UTG kann im Zeitbereich h\"aufig durch eine lineare Differentialgleichung (DGL)
beschrieben werden:
\[
  \sum_{m=0}^{n} a_m\,\frac{\dd^{m}x_a(t)}{\dd t^{m}}
  =
  \sum_{k=0}^{\ell} b_k\,\frac{\dd^{k}x_e(t)}{\dd t^{k}},
  \qquad a_n\neq 0.
\]

\paragraph{Beispiel (PT1).}
Ein Verz\"ogerungsglied erster Ordnung erf\"ullt
$T\,\dot x_a + x_a = K\,x_e$.
F\"ur einen Einheitssprung $x_e(t)=\sigma(t)$ ergibt sich eine exponentielle Ann\"aherung
an den Endwert $K$.

\subsection{Beispiele: typische \"UTG}
\begin{center}
  \begin{tabular}{@{}lll@{}}
    \toprule
    Name & Zeitbereich (DGL) & Bemerkung \\
    \midrule
    P-Glied   & $x_a = K_P\,x_e$ & proportional \\
    I-Glied   & $\dot x_a = K_I\,x_e$ & integratorisch \\
    D-Glied   & $x_a = K_D\,\dot x_e$ & ideal differenzierend \\
    PT1-Glied & $T\,\dot x_a + x_a = K\,x_e$ & Verz\"ogerung 1.\ Ordnung \\
    PT2-Glied & $T^2\,\ddot x_a + 2D T\,\dot x_a + x_a = K\,x_e$ & Verz\"ogerung 2.\ Ordnung \\
    DT1-Glied & $T\,\dot x_a + x_a = K_D\,\dot x_e$ & realisierbares D-Glied \\
    \bottomrule
  \end{tabular}
\end{center}

% kapitel/laplace-uebertragungsfunktion.tex
\section{Laplace-Transformation und \"Ubertragungsfunktion}
\subsection{Exponentialansatz und Laplace-Operator}
Viele f\"ur die Regelungstechnik relevante Signale lassen sich als (komplexe) Exponentialform schreiben:
\[
  x(t)=\hat X\,\mathrm{e}^{p t},
  \qquad
  p=\sigma+\jj\omega.
\]
F\"ur diesen Ansatz gilt:
\[
  \frac{\dd}{\dd t}x(t)=p\,x(t),\qquad
  \frac{\dd^2}{\dd t^2}x(t)=p^2\,x(t),\ \ldots
\]
Die Variable $p$ wirkt im Bildbereich als Differentialoperator.
Mit dem Laplace-Operator ist die Abbildung $\Laplace\{\cdot\}$ gemeint, die ein
Zeitbereichssignal $x(t)$ in seine Laplace-Transformierte $X(p)$ im $p$-Bereich
ueberfuehrt.
F\"ur ein kausales Signal $x(t)$:
\[
  X(p)=\mathcal{L}\{x(t)\}=\int_{0^-}^{\infty} x(t)\,\mathrm{e}^{-pt}\,\dd t.
\]
In der Regelungstechnik wird h\"aufig mit Null-Anfangsbedingungen gearbeitet, so dass Ableitungen
einfach in Multiplikationen \"ubergehen. Das sieht man aus der partiellen Integration:
\[
  \Laplace\{\dot x(t)\}=\int_{0^-}^{\infty}\dot x(t)\,\mathrm{e}^{-pt}\,\dd t
  =\bigl[x(t)\,\mathrm{e}^{-pt}\bigr]_{0^-}^{\infty}
  +p\int_{0^-}^{\infty}x(t)\,\mathrm{e}^{-pt}\,\dd t
  =-x(0^-)+pX(p),
\]
wobei der Randterm $\bigl[x(t)\,\mathrm{e}^{-pt}\bigr]_{0^-}^{\infty}$ der aus der partiellen
Integration stammende Bewertungsanteil an den Integrationsgrenzen ist. Fuer kausale, abklingende
Signale verschwindet der Anteil bei $t\to\infty$, weil $x(t)$ nicht schneller waechst als
$\mathrm{e}^{\sigma t}$ mit $\sigma<\Ree\{p\}$, und es bleibt der Anfangswertanteil $-x(0^-)$.
Analog gilt
\[
  \Laplace\{\ddot x(t)\}=p^2X(p)-p\,x(0^-)-\dot x(0^-).
\]
Bei Null-Anfangsbedingungen folgt damit $\Laplace\{\dot x(t)\}=pX(p)$ und
$\Laplace\{\ddot x(t)\}=p^2X(p)$.

\subsection{Laplace-Transformation des Dirac-Impulses}
Die Dirac-Impulsfunktion wirkt in Integralen als Abtaster (Sieb-Eigenschaft):
\[
  \int_{0^-}^{\infty}\delta(t)\,f(t)\,\dd t = f(0^-).
\]
Damit folgt direkt
\[
  \Laplace\{\delta(t)\}=\int_{0^-}^{\infty}\delta(t)\,\mathrm{e}^{-pt}\,\dd t=1.
\]
Fuer einen verschobenen Impuls gilt entsprechend
\[
  \Laplace\{\delta(t-t_0)\}=\int_{0^-}^{\infty}\delta(t-t_0)\,\mathrm{e}^{-pt}\,\dd t
  =\mathrm{e}^{-p t_0}\quad (t_0>0).
\]

\subsection{\"Ubertragungsfunktion (\"UTF)}
F\"ur ein LZI-\"UTG mit Null-Anfangsbedingungen gilt im Laplace-Bereich:
\[
  G(p)=\frac{X_a(p)}{X_e(p)}.
\]
Dabei sind $X_a(p)=\Laplace\{x_a(t)\}$ und $X_e(p)=\Laplace\{x_e(t)\}$ die Laplace-Transformationen
der Zeitfunktionen.
Die \"Ubertragungsfunktion $G(p)$ beschreibt damit die dynamische Beziehung zwischen Eingang und
Ausgang im $p$-Bereich und ist bei Null-Anfangsbedingungen allein durch das System bestimmt.
Sobald $G(p)$ bekannt ist, folgt f\"ur jede Anregung $X_a(p)=G(p)\,X_e(p)$.

Die zugeh\"orige Differentialgleichung stammt aus dem physikalischen Modell.

\paragraph{Beispiel (PT1, RC-Tiefpass).}
Ein einfacher RC-Tiefpass (Passfilter 1.\ Ordnung) mit Eingangsspannung $U_e(t)$ und
Ausgangsspannung $U_A(t)$ liefert mit dem Knotenstromsatz
\begin{center}
\begin{circuitikz}[european resistors]
  \draw
    (0,0) node[left]{$U_e(t)$}
    to[R, l=$R$] (3,0) node[circ]{}
    to[short] (4.5,0) node[right]{$U_A(t)$}
    (3,0) to[C, l=$C$] (3,-2)
    node[ground]{};
\end{circuitikz}
\end{center}
\[
  i_R=\frac{U_e(t)-U_A(t)}{R},\qquad i_C=C\,\dot U_A(t),\qquad i_R=i_C
\]
und damit
\[
  RC\,\dot U_A(t)+U_A(t)=U_e(t).
\]
Identifiziert man $x_a=U_A$, $x_e=U_e$, $T=RC$ und $K=1$, ergibt sich die PT1-DGL
\[
  T\,\dot x_a + x_a = K\,x_e.
\]
Im Laplace-Bereich folgt
\[
  (Tp+1)X_a(p)=K\,X_e(p)
  \quad\Longrightarrow\quad
  G(p)=\frac{K}{1+Tp}.
\]

\paragraph{Beispiel (Schrittanregung am PT1).}
F\"ur $x_e(t)=\sigma(t)$ gilt $X_e(p)=\frac{1}{p}$ und damit
\[
  X_a(p)=G(p)\,X_e(p)=\frac{K}{p(1+Tp)}.
\]
R\"ucktransformation liefert die Sprungantwort
\[
  x_a(t)=K\left(1-\mathrm{e}^{-t/T}\right)\sigma(t).
\]


% Step response of a PT1 element
\begin{figure}[tb]
  \centering
  \begin{tikzpicture}
    \begin{axis}[
      width=0.62\textwidth, height=0.38\textwidth,
      grid=both, grid style={line width=.1pt, draw=gray!20},
      major grid style={line width=.2pt, draw=gray!35},
      xlabel={$t/T$}, ylabel={$x_a(t)/K$},
      xmin=0, xmax=5,
      ymin=0, ymax=1.05,
      axis lines=left,
      tick label style={font=\small},
      label style={font=\small},
    ]
      \addplot[thick] table[x=t,y=y,col sep=comma] {data/pt1_step_K1_T1.csv};
      \addplot[dashed] coordinates {(0,1) (5,1)};
    \end{axis}
  \end{tikzpicture}
  \caption{Sprungantwort eines PT1-Glieds: $G(p)=\frac{K}{1+Tp}$, $x_a(t)=K(1-e^{-t/T})\sigma(t)$.}
  \label{fig:pt1_step}
\end{figure}


\paragraph{Beispiel (PT2 aus DGL).}
\[
  T^2\,\ddot x_a + 2DT\,\dot x_a + x_a = K\,x_e
  \quad\Longrightarrow\quad
  (T^2p^2+2DTp+1)X_a(p)=K\,X_e(p),
\]
\[
  G(p)=\frac{K}{1+2DTp+T^2p^2}.
\]

Weitere, umfangreichere \"Ubungsaufgaben finden sich im Anhang, siehe Abschnitt~\ref{app:aufgaben}.

\subsection{\"Ubersicht: DGL zu \"UTF}
\begin{center}
  \begin{tabular}{@{}lll@{}}
    \toprule
    Name & DGL & \"Ubertragungsfunktion $G(p)$ \\
    \midrule
    P-Glied   & $x_a = K_P\,x_e$ & $K_P$ \\
    D-Glied   & $x_a = K_D\,\dot x_e$ & $K_D\,p$ \\
    I-Glied   & $\dot x_a = K_I\,x_e$ & $\displaystyle\frac{K_I}{p}$ \\
    DT1-Glied & $T\,\dot x_a + x_a = K_D\,\dot x_e$ & $\displaystyle \frac{K_D\,p}{1+Tp}$ \\
    PT1-Glied & $T\,\dot x_a + x_a = K\,x_e$ & $\displaystyle \frac{K}{1+Tp}$ \\
    PT2-Glied & $T^2\,\ddot x_a + 2DT\,\dot x_a + x_a = K\,x_e$ & $\displaystyle \frac{K}{1+2DTp+T^2p^2}$ \\
    \bottomrule
  \end{tabular}
\end{center}

% kapitel/frequenzbereich.tex
\section{Beschreibung im Frequenzbereich}
\subsection{Frequenzgang}
F\"ur sinusf\"ormige Signale setzt man $p=\jj\omega$ und erh\"alt den Frequenzgang:
\[
  G(\jj\omega)=G(p)\big|_{p=\jj\omega}.
\]
Schreibweisen:
\[
  G(\jj\omega)=\Ree(\omega)+\jj\,\Imm(\omega)
  \quad\text{oder}\quad
  G(\jj\omega)=\lvert G(\jj\omega)\rvert\,\mathrm{e}^{\jj\varphi(\omega)}.
\]
Betrag und Phase:
\[
  \lvert G(\jj\omega)\rvert=\sqrt{\Ree(\omega)^2+\Imm(\omega)^2},
  \qquad
  \varphi(\omega)=\argg\!\bigl(G(\jj\omega)\bigr).
\]
Bei der Winkelbestimmung ist auf den Quadranten zu achten (praktisch: \texttt{atan2}).

\paragraph{Rechenregel (komplexe Br\"uche).}
F\"ur $G(\jj\omega)=\frac{Z(\omega)}{N(\omega)}$ mit $N(\omega)\neq 0$:
\[
  \frac{Z}{N}=\frac{Z\,\overline{N}}{N\,\overline{N}},
\]
woraus sich $\Ree,\Imm$ bequem bestimmen lassen.

\paragraph{Beispiel (PT1, $K=1$).}
Mit $G(\jj\omega)=\frac{1}{1+\jj\omega T}$ folgt
\[
  G(\jj\omega)=\frac{1-\jj\omega T}{1+(\omega T)^2},
  \quad
  \Ree(\omega)=\frac{1}{1+(\omega T)^2},\ 
  \Imm(\omega)=\frac{-\omega T}{1+(\omega T)^2}.
\]

\subsection{Ortskurve}
Die Ortskurve ist der geometrische Ort aller Werte von $G(\jj\omega)$ in der komplexen Ebene,
wenn $\omega$ von $0$ bis $\infty$ l\"auft.

% kapitel/bode-diagramme.tex
\section{Bode-Diagramme}
\subsection{Amplitude in Dezibel}
\[
  L(\omega)=20\log_{10}\!\bigl(\lvert G(\jj\omega)\rvert\bigr)\ \si{dB}.
\]
Wichtiger Vorteil:
\[
  G(\jj\omega)=G_1(\jj\omega)\,G_2(\jj\omega)
  \quad\Rightarrow\quad
  L(\omega)=L_1(\omega)+L_2(\omega).
\]
Analog addieren sich Phasen:
\[
  \varphi(\omega)=\varphi_1(\omega)+\varphi_2(\omega).
\]

\subsection{Beispiel: PT1-Glied}
\[
  G(\jj\omega)=\frac{K}{1+\jj\omega T}.
\]
Damit:
\[
  \lvert G(\jj\omega)\rvert=\frac{K}{\sqrt{1+(\omega T)^2}},
  \qquad
  \varphi(\omega)=-\arctan(\omega T).
\]
Asymptotisch:
\begin{itemize}[itemsep=2pt]
  \item $\omega\ll 1/T$: $\lvert G\rvert\approx K$ (0\,dB-Steigung), $\varphi\approx 0^\circ$.
  \item $\omega\gg 1/T$: $\lvert G\rvert\approx \frac{K}{\omega T}$ (Steigung \SI{-20}{dB/dec}), $\varphi\approx -90^\circ$.
\end{itemize}


% Bode diagram of a PT1 element
\begin{figure}[tb]
  \centering
  \begin{tikzpicture}
    \begin{groupplot}[
      group style={group size=1 by 2, vertical sep=0.9cm},
      width=0.78\textwidth, height=0.34\textwidth,
      grid=both, grid style={line width=.1pt, draw=gray!20},
      major grid style={line width=.2pt, draw=gray!35},
      xmode=log,
      xmin=1e-2, xmax=1e2,
      axis lines=left,
      tick label style={font=\small},
      label style={font=\small},
    ]
      \nextgroupplot[ylabel={$L(\omega)$ / dB}, title={Bode-Diagramm (PT1, $K=1,\,T=1$)}]
        \addplot[thick] table[x=w,y=mag_db,col sep=comma] {data/pt1_bode_K1_T1.csv};
        \addplot[dashed] coordinates {(1,0) (1e2,-40)};
        \addplot[dashed] coordinates {(1e-2,0) (1,0)};
      \nextgroupplot[xlabel={$\omega T$}, ylabel={$\varphi(\omega)$ / $^\circ$}, ymin=-100, ymax=10]
        \addplot[thick] table[x=w,y=phase_deg,col sep=comma] {data/pt1_bode_K1_T1.csv};
        \addplot[dashed] coordinates {(1,-45) (1,-45)};
        \addplot[dashed] coordinates {(1,-90) (1e2,-90)};
    \end{groupplot}
  \end{tikzpicture}
  \caption{Bode-Diagramm des PT1-Glieds. Die gestrichelten Linien zeigen die asymptotischen Näherungen (0~dB bzw.\ $-20$~dB/dec, Phase gegen $-90^\circ$).}
  \label{fig:pt1_bode}
\end{figure}


\paragraph{Beispiel (Verst\"arkung).}
Wird $K$ von $1$ auf $10$ erh\"oht, verschiebt sich der Amplitudengang um $20\,\mathrm{dB}$
nach oben; die Phasenlage bleibt unver\"andert.

\FloatBarrier


% kapitel/typische-lineare-uebertragungsglieder.tex
\section{D\"ampfungsph\"anomene und schwingungsf\"ahige lineare \"UTG}
\label{sec:ueberschwingen-eigenfrequenz}

\subsection{Worum geht es?}
In vielen technischen Systemen (Regelkreise, Filter, mechanische Strukturen, elektrische Netzwerke)
tritt nach einer pl\"otzlichen \"Anderung des Eingangssignals (z.\,B.\ Sprung) eine dynamische
Antwort auf. Dabei beobachtet man h\"aufig:

\begin{itemize}
  \item \textbf{Schwingen/Oszillationen:} Das Ausgangssignal zeigt periodische Anteile.
  \item \textbf{\"Uberschwingen (Overshoot):} Der Ausgang \"uberschreitet kurzzeitig den sp\"ateren Endwert.
  \item \textbf{Eigenfrequenzen:} Frequenzen, bei denen das System ohne \"au\ss ere Anregung (frei) schwingt
        bzw.\ besonders stark auf Anregungen reagiert.
\end{itemize}

Wichtig: Ein \emph{PT1-System} (z.\,B.\ einfacher RC-Tiefpass) besitzt nur einen Pol und zeigt
typischerweise \emph{kein} \"Uberschwingen. \"Uberschwingen ist ein Kennzeichen von Systemen
\emph{mindestens zweiter Ordnung} (z.\,B.\ RLC-Netzwerke, PT2, viele Regelkreise).

\subsection{D\"ampfungsph\"anomene am Beispiel des PT2-Glieds}
\[
  G(p)=\frac{K}{1+2DTp+T^2p^2}.
\]
Eine h\"aufige, normierte Schreibweise ist
\begin{equation}
  G(s) = \frac{\omega_0^2}{s^2 + 2\zeta\omega_0 s + \omega_0^2}.
  \label{eq:pt2}
\end{equation}
\begin{itemize}[itemsep=2pt]
  \item $D$ \dots\ D\"ampfungsma\ss\ (entspricht $\zeta$).
  \item $T$ \dots\ Zeitkonstante (in dieser Normierung entspricht $\omega_0=\frac{1}{T}$ der unged\"ampften Eigenkreisfrequenz).
\end{itemize}
\paragraph{G\"utefaktor.}
H\"aufig wird die D\"ampfung \"uber den \textbf{G\"utefaktor} beschrieben:
\[
  Q = \frac{1}{2D}=\frac{1}{2\zeta}.
\]
Gro\ss es $Q$ bedeutet geringe D\"ampfung $\Rightarrow$ ausgepr\"agtere Resonanzspitze und tendenziell mehr \"Uberschwingen; kleines $Q$ steht f\"ur starke D\"ampfung und rasches Abklingen.


% Magnitude response of PT2 for different damping ratios D
\begin{figure}[tb]
  \centering
  \begin{tikzpicture}
    \begin{axis}[
      width=0.78\textwidth, height=0.45\textwidth,
      grid=both, grid style={line width=.1pt, draw=gray!20},
      major grid style={line width=.2pt, draw=gray!35},
      xmode=log,
      xlabel={$\omega T$}, ylabel={$L(\omega)$ / dB},
      xmin=1e-2, xmax=1e2,
      ymin=-60, ymax=20,
      axis lines=left,
      legend style={font=\small, at={(0.02,0.02)}, anchor=south west},
      tick label style={font=\small},
      label style={font=\small},
    ]
      \addplot[thick] table[x=w,y=D0.2,col sep=comma] {data/pt2_mag_bode_T1_K1_multiD.csv};
      \addlegendentry{$D=0{,}2$}
      \addplot[thick, densely dashed] table[x=w,y=D0.5,col sep=comma] {data/pt2_mag_bode_T1_K1_multiD.csv};
      \addlegendentry{$D=0{,}5$}
      \addplot[thick, dashdotted] table[x=w,y=D0.8,col sep=comma] {data/pt2_mag_bode_T1_K1_multiD.csv};
      \addlegendentry{$D=0{,}8$}
      \addplot[thick, dotted] table[x=w,y=D1.2,col sep=comma] {data/pt2_mag_bode_T1_K1_multiD.csv};
      \addlegendentry{$D=1{,}2$}
      \addplot[dashed] coordinates {(1,0) (1,0)};
    \end{axis}
  \end{tikzpicture}
  \caption{Betragsfrequenzgang eines normierten PT2-Glieds ($T=1$, $K=1$). Kleine Dämpfung erzeugt eine Resonanzüberhöhung (Peak im Betrag), die im Zeitbereich typischerweise mit starkem Überschwingen korreliert.}
  \label{fig:pt2_mag_multiD}
\end{figure}


\paragraph{Kurze Herleitung der Normform.}
Ausgehend von $G(p)=\dfrac{K}{1+2DTp+T^2p^2}$ wird der Nenner durch $T^2$ normiert:
\[
  G(s)=\frac{K}{1+2DTs+T^2s^2}
  =\frac{\tfrac{K}{T^2}}{s^2+\tfrac{2D}{T}s+\tfrac{1}{T^2}}.
\]
Durch Gleichsetzen der Nennerkoeffizienten mit der Standardform $s^2+2\zeta\omega_0 s+\omega_0^2$ folgt
\[
  \omega_0=\frac{1}{T}, \qquad \zeta = D, \qquad \frac{K}{T^2}=K\omega_0^2.
\]
Setzt man $K=1$ (normierte statische Verst\"arkung), ergibt sich direkt \eqref{eq:pt2}.

\paragraph{Sprungantwort des Standard-PT2.}
F\"ur den Einheitssprung gilt $X_e(s)=1/s$. Damit
\[
  Y(s)=G(s)X_e(s)=\frac{\omega_0^2}{s\left(s^2+2D\omega_0 s+\omega_0^2\right)}.
\]
Schreibe den Zaehler als Differenz:
\begin{align*}
  \frac{\omega_0^2}{s\left(s^2+2D\omega_0 s+\omega_0^2\right)}
  &=\frac{\left(s^2+2D\omega_0 s+\omega_0^2\right)-\left(s^2+2D\omega_0 s\right)}
          {s\left(s^2+2D\omega_0 s+\omega_0^2\right)}\\
  &=\frac{1}{s}-\frac{s^2+2D\omega_0 s}{s\left(s^2+2D\omega_0 s+\omega_0^2\right)}\\
  &=\frac{1}{s}-\frac{s+2D\omega_0}{s^2+2D\omega_0 s+\omega_0^2}.
\end{align*}
Quadratische Ergaenzung im Nenner (Schritt fuer Schritt):
\begin{align*}
  s^2+2D\omega_0 s+\omega_0^2
  &=\bigl(s^2+2D\omega_0 s+D^2\omega_0^2\bigr)
    +\omega_0^2-D^2\omega_0^2\\
  &=(s+D\omega_0)^2+\bigl(1-D^2\bigr)\omega_0^2.
\end{align*}
Zaehler in zwei Teile zerlegen:
\[
  s+2D\omega_0=(s+D\omega_0)+D\omega_0.
\]
Definiere nun $\omega_d=\omega_0\sqrt{1-D^2}$, also $(1-D^2)\omega_0^2=\omega_d^2$. Damit:
\[
  Y(s)=\frac{1}{s}-\frac{s+D\omega_0}{(s+D\omega_0)^2+\omega_d^2}
       -\frac{D\omega_0}{(s+D\omega_0)^2+\omega_d^2}.
\]
Ruecktransformation mit den bekannten Paaren (Herleitungen der Ruecktransformationen sind nicht Teil dieses Skripts):
\begin{align}
  \Laplace^{-1}\!\left\{\frac{1}{s}\right\}&=1, \tag{L1}\label{eq:laplace_pair_L1}\\
  \Laplace^{-1}\!\left\{\frac{s+a}{(s+a)^2+\omega_d^2}\right\}&=\mathrm{e}^{-a t}\cos(\omega_d t), \tag{L2}\label{eq:laplace_pair_L2}\\
  \Laplace^{-1}\!\left\{\frac{\omega_d}{(s+a)^2+\omega_d^2}\right\}&=\mathrm{e}^{-a t}\sin(\omega_d t). \tag{L3}\label{eq:laplace_pair_L3}
\end{align}
Im letzten Ausdruck stehen zwei Bruchterme mit Quadratterm im Nenner.
Der Zaehler $s+D\omega_0$ passt direkt zum Cosinus-Paar in \eqref{eq:laplace_pair_L2}.
Fuer den zweiten Bruchterm braucht man das Sinus-Paar in \eqref{eq:laplace_pair_L3}, dort steht $\omega_d$ im Zaehler.
Wir nutzen dabei die Linearitaet der Ruecktransformation:
\[
  \Laplace^{-1}\!\{c\,F(s)\}=c\,\Laplace^{-1}\!\{F(s)\}.
\]
Daher wird der Bruchterm als Konstante mal Standardform geschrieben:
\[
  \frac{D\omega_0}{(s+a)^2+\omega_d^2}
  =\frac{D\omega_0}{\omega_d}\,\frac{\omega_d}{(s+a)^2+\omega_d^2}.
\]
Herleitungen zu bekannten Fourier-Transformationspaaren finden sich z.\,B.\ in
\href{https://en.wikipedia.org/wiki/Fourier_transform#Table_of_important_Fourier_transforms}{Fourier-Transformationspaaren}.
Setze $a=D\omega_0$. Mit \eqref{eq:laplace_pair_L3} folgt:
\[
  \Laplace^{-1}\!\left\{\frac{D\omega_0}{(s+a)^2+\omega_d^2}\right\}
  =\frac{D\omega_0}{\omega_d}\,\mathrm{e}^{-a t}\sin(\omega_d t).
\]
Da $\omega_d=\omega_0\sqrt{1-D^2}$ gilt, folgt
\[
  \frac{D\omega_0}{\omega_d}=\frac{D}{\sqrt{1-D^2}}.
\]
Damit ergibt sich
\begin{equation}
  x_a(t)=1-\mathrm{e}^{-D\omega_0 t}\!\left(\cos(\omega_d t)+\frac{D}{\sqrt{1-D^2}}\sin(\omega_d t)\right),
  \qquad \omega_d=\omega_0\sqrt{1-D^2}.
  \label{eq:pt2_step_response}
\end{equation}

\paragraph{Rechenweg f\"ur einen elektrischen Schwingkreis.}
% Generic RLC series circuit as PT2 low-pass (non-floating)
\refstepcounter{figure}%
\begin{center}
  \begin{circuitikz}[american voltages, scale=1]
    \draw
      (0,0) to[sV, l=$u_\mathrm{e}$] (0,3)
      (0,3) to[R, l=$R$] (2.5,3)
      to[L, l=$L$] (5,3)
      to[C, l=$C$, v^<=$u_C$] (5,0)
      (5,0) -- (0,0);
  \end{circuitikz}
\end{center}
\vspace{-0.4em}
\noindent\textbf{Abbildung \thefigure:} Reihenschwingkreis als PT2-Tiefpass ($u_C$ als Ausgang). Die Parameter $R$, $L$, $C$ werden mit $T=\sqrt{LC}$ und $D=\tfrac{R}{2}\sqrt{\tfrac{C}{L}}$ in die Normform eingesetzt.
\label{fig:pt2_schwingkreis}


Betrachte einen Reihenschwingkreis aus $R$, $L$, $C$ (Ausgangsspannung am Kondensator, Tiefpass-PT2). Mit $Z_R=R$, $Z_L=sL$, $Z_C=\tfrac{1}{sC}$ liefert die Spannungsteilung
\[
  G(s)=\frac{U_C(s)}{U_E(s)}=\frac{Z_C}{Z_R+Z_L+Z_C}
      =\frac{\tfrac{1}{sC}}{R+sL+\tfrac{1}{sC}}
      =\frac{1}{LC\,s^2 + RC\,s + 1}.
\]
Vergleiche den Nenner mit $1+2DTs+T^2s^2$:
\[
  T^2=LC,\qquad 2DT=RC \quad\Rightarrow\quad T=\sqrt{LC},\quad D=\frac{R}{2}\sqrt{\frac{C}{L}}.
\]
Damit
\[
  \omega_0=\frac{1}{T}=\frac{1}{\sqrt{LC}},\qquad Q=\frac{1}{2D}=\sqrt{\frac{L}{C}}\frac{1}{R}.
\]
Analog verf\"ahrt man bei mechanischen Schwingkreisen (Masse--Feder--D\"ampfer), indem die Koeffizienten der Bewegungsgleichung mit der PT2-Normform abgeglichen werden.

Die Polstellen von \eqref{eq:pt2} sind
\[
  s_{1,2} = -\zeta\omega_0 \pm j\,\omega_0\sqrt{1-\zeta^2}.
\]

\paragraph{Zeitpunkt des ersten Maximums.}
Ein Maximum liegt an einer Stelle mit $\dot x_a(t)=0$. Daher leiten wir die Sprungantwort
(vgl. \eqref{eq:pt2_step_response}) ab:
\[
  \dot x_a(t)=\frac{\omega_0}{\sqrt{1-D^2}}\,\mathrm{e}^{-D\omega_0 t}\,\sin(\omega_d t).
\]
Damit gilt $\dot x_a(t)=0$ f\"ur $t=k\pi/\omega_d$; der erste positive Extremwert liegt bei $k=1$.
Die Zeit bis zum ersten Peak (peak time) ist somit
\begin{equation}
  t_p = \frac{\pi}{\omega_d} = \frac{\pi}{\omega_0\sqrt{1-D^2}}.
  \label{eq:tp}
\end{equation}

\paragraph{Einschwingzeit (Faustregel).}
Die Schwingungsamplitude folgt der H\"ulle $\mathrm{e}^{-D\omega_0 t}$.
Als 2\,\%-Einschwingzeit wird h\"aufig die Faustregel verwendet:
\begin{equation}
  t_s \approx \frac{4}{D\,\omega_0}=\frac{4T}{D}.
  \label{eq:ts}
\end{equation}
(Die N\"aherung ist am besten im mittleren D\"ampfungsbereich und dient der schnellen Absch\"atzung.)

\subsubsection{\"Uberschwingen bei der Sprungantwort}
Betrachte einen Einheitssprung $x_e(t)=\sigma(t)$ am Eingang eines stabilen Systems der Form
\eqref{eq:pt2}. F\"ur $0<D<1$ zeigt die Sprungantwort typischerweise ein Maximum oberhalb des Endwerts.

Die Sprungantwort lautet \eqref{eq:pt2_step_response}.

\paragraph{Differentialgleichung des Standard-PT2.}
Aus \eqref{eq:pt2} folgt mit $G(s)=\dfrac{Y(s)}{X_e(s)}$ und damit $Y(s)=G(s)X_e(s)$:
\[
  Y(s)=\frac{\omega_0^2}{s^2+2D\omega_0 s+\omega_0^2}\,X_e(s).
\]
Multiplikation mit dem Nenner liefert
\[
  \left(s^2+2D\omega_0 s+\omega_0^2\right)Y(s)=\omega_0^2\,X_e(s).
\]
R\"ucktransformation (bei Null-Anfangsbedingungen) liefert im Zeitbereich
\[
  \ddot x_a(t)+2D\omega_0 \dot x_a(t)+\omega_0^2 x_a(t)=\omega_0^2 x_e(t).
\]
F\"ur den Einheitssprung gilt $x_e(t)=\sigma(t)$, also
\[
  \ddot x_a+2D\omega_0 \dot x_a+\omega_0^2 x_a=\omega_0^2.
\]

\paragraph{D\"ampfungsf\"alle.}
\begin{itemize}[itemsep=2pt]
  \item $D<0$ (\emph{negative D\"ampfung}): Energiezufuhr statt Verlust, Schwingungen wachsen an (instabil).
  \item $0<D<1$ (\emph{unterd\"ampft}): komplex konjugierte Pole, schwingende Sprungantwort mit \"Uberschwingen.
  \item $D=1$ (\emph{kritisch ged\"ampft}): Doppelpol, schnell ohne \"Uberschwingen (Grenzfall).
  \item $D>1$ (\emph{\"uberd\"ampft}): zwei reelle Pole, monotone Sprungantwort ohne \"Uberschwingen.
\end{itemize}

% Step response of PT2 for different damping ratios D
\begin{figure}[tb]
  \centering
  \begin{tikzpicture}
    \begin{axis}[
      width=0.78\textwidth, height=0.45\textwidth,
      grid=both, grid style={line width=.1pt, draw=gray!20},
      major grid style={line width=.2pt, draw=gray!35},
      xlabel={$t/T$}, ylabel={$x_a(t)/K$},
      xmin=0, xmax=20,
      ymin=-0.05, ymax=1.8,
      axis lines=left,
      legend style={font=\small, at={(0.98,0.02)}, anchor=south east},
      tick label style={font=\small},
      label style={font=\small},
    ]
      \addplot[thick] table[x=t,y=D0.2,col sep=comma] {data/pt2_step_T1_K1_multiD.csv};
      \addlegendentry{$D=0{,}2$}
      \addplot[thick, densely dashed] table[x=t,y=D0.5,col sep=comma] {data/pt2_step_T1_K1_multiD.csv};
      \addlegendentry{$D=0{,}5$}
      \addplot[thick, dashdotted] table[x=t,y=D0.8,col sep=comma] {data/pt2_step_T1_K1_multiD.csv};
      \addlegendentry{$D=0{,}8$}
      \addplot[thick, dotted] table[x=t,y=D1.0,col sep=comma] {data/pt2_step_T1_K1_multiD.csv};
      \addlegendentry{$D=1{,}0$}
      \addplot[thick] table[x=t,y=D1.2,col sep=comma] {data/pt2_step_T1_K1_multiD.csv};
      \addlegendentry{$D=1{,}2$}
      \addplot[dashed] coordinates {(0,1) (20,1)};
    \end{axis}
  \end{tikzpicture}
  \caption{Sprungantwort eines normierten PT2-Glieds ($T=1$, $K=1$) f\"ur unterschiedliche D\"ampfungsma\ss e $D$. Unterd\"ampfung ($D<1$) f\"uhrt zu \"Uberschwingen und Schwingungen; bei $D\ge1$ verschwindet das \"Uberschwingen.}
  \label{fig:pt2_step_multiD}
\end{figure}

\FloatBarrier

\textbf{\"Uberschwingma\ss} (relative peak overshoot) $M_p$:
\[
  M_p = \frac{x_{a,\text{max}}-\lim_{t\to\infty}x_a(t)}{\lim_{t\to\infty}x_a(t)}.
\]
Oft wird es in Prozent angegeben: $M_p[\%] = 100\cdot M_p$.

F\"ur das Standard-PT2 gilt (bei Einheitssprung, $0<D<1$):
\begin{equation}
  M_p=\exp\!\left(-\frac{D\pi}{\sqrt{1-D^2}}\right).
  \label{eq:overshoot}
\end{equation}
Herleitung (kurz): Mit \eqref{eq:pt2_step_response} und $t_p=\pi/\omega_d$ (siehe \eqref{eq:tp}) folgt
\[
  x_a(t_p)=1-\mathrm{e}^{-D\omega_0 t_p}\cos(\pi)=1+\mathrm{e}^{-D\omega_0 t_p},
\]
und somit
\[
  M_p=\frac{x_a(t_p)-1}{1}=\mathrm{e}^{-D\omega_0 t_p}
  =\exp\!\left(-D\omega_0\cdot\frac{\pi}{\omega_0\sqrt{1-D^2}}\right)
  =\exp\!\left(-\frac{D\pi}{\sqrt{1-D^2}}\right).
\]
Umgekehrt:
\[
  \ln(M_p)=-\frac{D\pi}{\sqrt{1-D^2}}
  \ \Rightarrow\ 
  \ln^2(M_p)=\frac{D^2\pi^2}{1-D^2}
  \ \Rightarrow\ 
  D^2=\frac{\ln^2(M_p)}{\pi^2+\ln^2(M_p)}.
\]
\[
  D=\frac{-\ln(M_p)}{\sqrt{\pi^2+\ln^2(M_p)}}.
\]

\paragraph{Logarithmisches Daempfungsdekrement.}
F\"ur eine unterd\"ampfte PT2-Sprungantwort bezeichnen $x_k$ und $x_{k+1}$ zwei aufeinanderfolgende
Maxima, $x_\infty$ den Endwert. Dann ist das logarithmische Daempfungsdekrement
\[
  \Lambda=\ln\!\left(\frac{x_k-x_\infty}{x_{k+1}-x_\infty}\right)
  =\frac{2\pi D}{\sqrt{1-D^2}}.
\]
Herleitung (kurz): F\"ur den normierten PT2 und den Einheitssprung $x_e(t)=\sigma(t)$ gilt im Zeitbereich
\[
  \ddot x_a + 2D\omega_0 \dot x_a + \omega_0^2 x_a = \omega_0^2.
\]
Homogene Gleichung: Ansatz $x_h=\mathrm{e}^{s t}$ liefert die charakteristische Gleichung
\[
  s^2+2D\omega_0 s+\omega_0^2=0,
\]
deren L\"osungen $s_{1,2}=-D\omega_0\pm \jj\,\omega_d$ mit $\omega_d=\omega_0\sqrt{1-D^2}$ sind.
Zu jedem Eigenwert $s_i$ geh\"ort ein Summand $\mathrm{e}^{s_i t}$, daher
\[
  x_h(t)=C_1\,\mathrm{e}^{s_1 t}+C_2\,\mathrm{e}^{s_2 t}.
\]
Einsetzen von $s_{1,2}$ ergibt
\[
  x_h(t)=C_1\,\mathrm{e}^{(-D\omega_0+\jj\omega_d)t}+C_2\,\mathrm{e}^{(-D\omega_0-\jj\omega_d)t}.
\]
Aus $\mathrm{e}^{(a+b)t}=\mathrm{e}^{a t}\mathrm{e}^{b t}$ folgt durch Ausklammern von
$\mathrm{e}^{-D\omega_0 t}$:
\[
  x_h(t)=\mathrm{e}^{-D\omega_0 t}\bigl(C_1\,\mathrm{e}^{\jj\omega_d t}+C_2\,\mathrm{e}^{-\jj\omega_d t}\bigr).
\]
Mit der Euler-Formel fasst man die komplexen Exponentialanteile zu Cosinus und Sinus zusammen:
\[
  x_h(t)=\mathrm{e}^{-D\omega_0 t}\bigl(A\cos(\omega_d t)+B\sin(\omega_d t)\bigr).
\]
Dabei gilt explizit
\[
  \mathrm{e}^{\jj\omega_d t}=\cos(\omega_d t)+\jj\sin(\omega_d t),\qquad
  \mathrm{e}^{-\jj\omega_d t}=\cos(\omega_d t)-\jj\sin(\omega_d t),
\]
so dass sich die reelle Linearkombination in Cosinus- und Sinusanteile zerlegt.
Die Gesamtl\"osung ist die Summe aus homogener und partikul\"arer L\"osung:
\[
  x_a(t)=x_h(t)+x_p(t).
\]
Die partikul\"are L\"osung ist der station\"are Anteil und gilt f\"ur alle Zeiten als
$x_a(t)=x_p(t)+x_h(t)$. Bei stabilem System verschwindet der transiente Anteil
$x_h(t)$ f\"ur $t\to\infty$, sodass dann $x_a(t)\to x_p(t)$.
Rechte Seite ist konstant $\Rightarrow$ der station\"are Anteil ist konstant; daraus folgt
$x_p(t)=x_\infty=1$.
Damit gilt $x_a(t)-x_\infty=x_h(t)$. Die Linearkombination l\"asst sich zu einer Sinusform
zusammenfassen, denn
\[
  A\cos(\omega_d t)+B\sin(\omega_d t)=R\sin(\omega_d t+\varphi).
\]
Beweis: Setze $R=\sqrt{A^2+B^2}$ und w\"ahle $\varphi$ mit
\[
  R\sin\varphi=A,\qquad R\cos\varphi=B \quad (\tan\varphi=A/B).
\]
Dann gilt
\[
  R\sin(\omega_d t+\varphi)
  =R\bigl(\sin(\omega_d t)\cos\varphi+\cos(\omega_d t)\sin\varphi\bigr)
  =R\sin(\omega_d t)\cos\varphi+R\cos(\omega_d t)\sin\varphi
  =B\sin(\omega_d t)+A\cos(\omega_d t).
\]
Daher gilt fuer die Abweichung
\[
  x_a(t)-x_\infty=\mathrm{e}^{-D\omega_0 t}\bigl(A\cos(\omega_d t)+B\sin(\omega_d t)\bigr),
\]
und damit kann man schreiben
\[
  x_a(t)-x_\infty = R\,\mathrm{e}^{-D\omega_0 t}\sin(\omega_d t+\varphi).
\]
Sei $t_k$ der Zeitpunkt des $k$-ten Maximums der Sprungantwort, dann ist
$x_k=x_a(t_k)$ und $x_\infty=\lim_{t\to\infty}x_a(t)$ der Endwert. Aus
\[
  x_a(t)-x_\infty=R\,\mathrm{e}^{-D\omega_0 t}\sin(\omega_d t+\varphi)
\]
folgt an einem Maximum $\sin(\omega_d t_k+\varphi)=1$ und damit
\[
  x_k-x_\infty=R\,\mathrm{e}^{-D\omega_0 t_k}.
\]
Der Sinus hat die Periode $T_d=\frac{2\pi}{\omega_d}$, daher liegen aufeinanderfolgende Maxima
um genau eine Periode auseinander: $t_{k+1}-t_k=T_d$. Damit
\[
  \frac{x_k-x_\infty}{x_{k+1}-x_\infty}
  =\frac{R\,\mathrm{e}^{-D\omega_0 t_k}}{R\,\mathrm{e}^{-D\omega_0 t_{k+1}}}
  =\exp\!\bigl(D\omega_0(t_{k+1}-t_k)\bigr)
  =\exp\!\left(D\omega_0\frac{2\pi}{\omega_d}\right),
\]
und mit $\omega_d=\omega_0\sqrt{1-D^2}$ folgt
\[
  \Lambda=\ln\!\left(\frac{x_k-x_\infty}{x_{k+1}-x_\infty}\right)
  =\frac{2\pi D}{\sqrt{1-D^2}}.
\]
Zusammenhang zum \"Uberschwingma\ss: F\"ur den normierten Endwert $x_\infty=1$ gilt
\[
  M_p=\exp\!\left(-\frac{D\pi}{\sqrt{1-D^2}}\right)=\exp\!\left(-\frac{\Lambda}{2}\right),
  \qquad
  \Lambda=2\ln\!\left(\frac{1}{M_p}\right).
\]
Warum logarithmisch? Die H\"ullkurve der Schwingungsamplitude f\"allt exponentiell, d.\,h.\ die
Verh\"altnisse aufeinanderfolgender Maxima sind konstant. Der Logarithmus macht diese
multiplikative Abnahme additiv und unabh\"angig von der Anfangsamplitude. Vorteil: $D$ l\"asst
sich direkt aus zwei (oder mehreren gemittelten) Peaks robust bestimmen.

% Logarithmisches Daempfungsdekrement an einer unterdaempften PT2-Sprungantwort
\begin{figure}[!ht]
  \centering
  \begin{tikzpicture}
    \pgfmathsetmacro{\damp}{0.2}
    \pgfmathsetmacro{\omegad}{sqrt(1-\damp*\damp)}
    \pgfmathsetmacro{\tpeakone}{pi/\omegad}
    \pgfmathsetmacro{\tpeaktwo}{\tpeakone + 2*pi/\omegad}
    \pgfmathsetmacro{\mpeak}{exp(-\damp*pi/\omegad)}
    \pgfmathsetmacro{\xone}{1 + \mpeak}
    \pgfmathsetmacro{\xtwo}{1 + \mpeak*exp(-2*pi*\damp/\omegad)}
    \begin{axis}[
      width=0.78\textwidth, height=0.36\textwidth,
      grid=both, grid style={line width=.1pt, draw=gray!20},
      major grid style={line width=.2pt, draw=gray!35},
      xlabel={$t$}, ylabel={$x(t)$},
      xmin=0, xmax=11,
      ymin=-0.2, ymax=1.7,
      axis lines=left,
      tick label style={font=\small},
      label style={font=\small},
      trig format=rad
    ]
      \addplot[thick, domain=0:11, samples=400]
        {1 - exp(-\damp*x)*(cos(\omegad*x) + (\damp/\omegad)*sin(\omegad*x))};
      \addplot[dashed] coordinates {(0,1) (11,1)};
      % Daempfungskurve zwischen zwei Maxima (Envelope der Peak-Abstaende)
      \addplot[densely dashed, domain=\tpeakone:\tpeaktwo]
        {1 + \mpeak*exp(-\damp*(x-\tpeakone))};
      \addplot[only marks, mark=*] coordinates {(\tpeakone,\xone) (\tpeaktwo,\xtwo)};
      \draw[->] (axis cs:\tpeakone,\xone) -- (axis cs:\tpeaktwo,\xtwo)
        node[midway, above] {$\Lambda=\ln\!\left(\frac{x_k-x_\infty}{x_{k+1}-x_\infty}\right)$};
      \node[anchor=west] at (axis cs:6.2,1.05) {$x_\infty$};
      \node[anchor=west] at (axis cs:6.2,1.45) {D\"ampfungskurve};
    \end{axis}
  \end{tikzpicture}
  \caption{Unterd\"ampfte PT2-Sprungantwort: Das logarithmische Daempfungsdekrement $\Lambda$ ergibt sich aus zwei aufeinanderfolgenden Maxima der Abweichung von $x_\infty$.}
  \label{fig:log_decrement}
\end{figure}


\paragraph{Beispiel (\"Uberschwingen).}
F\"ur $D=0{,}5$ ergibt sich
$M_p=\exp\!\left(-\frac{0{,}5\pi}{\sqrt{1-0{,}25}}\right)\approx 0{,}16$,
also etwa $16\,\%$ \"Uberschwingen.

% Overshoot vs damping ratio
\begin{figure}[tb]
  \centering
  \begin{tikzpicture}
    \begin{axis}[
      width=0.70\textwidth, height=0.42\textwidth,
      grid=both, grid style={line width=.1pt, draw=gray!20},
      major grid style={line width=.2pt, draw=gray!35},
      xlabel={Dämpfungsmaß $D$}, ylabel={$M_p$ / \%},
      xmin=0, xmax=1,
      ymin=0, ymax=110,
      axis lines=left,
      tick label style={font=\small},
      label style={font=\small},
    ]
      \addplot[thick] table[x=D,y=Mp_percent,col sep=comma] {data/pt2_overshoot_vs_damping.csv};
      \addplot[dashed] coordinates {(0.5,0) (0.5,110)};
      \node[anchor=west, font=\small] at (axis cs:0.51,70) {$D=0{,}5 \Rightarrow M_p\approx16\%$};
    \end{axis}
  \end{tikzpicture}
  \caption{Relatives Überschwingen $M_p=\exp\!\left(-\frac{D\pi}{\sqrt{1-D^2}}\right)$ (unterdämpfter Fall $0<D<1$). Mit wachsender Dämpfung sinkt das Überschwingen stark.}
  \label{fig:overshoot_vs_damping}
\end{figure}


\subsubsection{Eigenfrequenz vs.\ Resonanzfrequenz}
Die \textbf{Eigenfrequenz} beschreibt das freie Schwingverhalten (durch die Pole bestimmt).
Die \textbf{Resonanzfrequenz} beschreibt dagegen die Frequenz, bei der die \emph{erzwungene} Antwort
(z.\,B.\ Sinusanregung) maximal wird.

F\"ur das System \eqref{eq:pt2} zeigt die Betragsfrequenzgangkurve $|G(j\omega)|$ nur dann eine ausgepr\"agte
Resonanzspitze, wenn die D\"ampfung klein genug ist (typisch $D < 1/\sqrt{2}$).
Dann gilt n\"aherungsweise:
\begin{equation}
  \omega_r = \omega_0\sqrt{1-2D^2}
  \qquad (D < 1/\sqrt{2}).
  \label{eq:omega_r}
\end{equation}
Man sieht: $\omega_r$ liegt dann \emph{unter} $\omega_0$.
Kleine D\"ampfung erzeugt eine Resonanzspitze im Betragsfrequenzgang in der N\"ahe von
$\omega\approx\omega_0$. Diese Resonanz\"uberh\"ohung korreliert h\"aufig mit starkem
\"Uberschwingen im Zeitbereich.

% kapitel/07-blockdiagramm-umformung.tex
\section{Blockdiagramme}
\subsection{Grundidee und Motivation}
Blockdiagramme beschreiben den Signalfluss in Regelkreisen: Bl\"ocke stehen f\"ur Teilsysteme (Strecke, Regler, Sensor), Pfeile f\"ur Signale. Sie sind damit mehr als ein Werkzeug zum Umformen --- sie sind der ``Schaltplan'' der Regelstrecke und zeigen sofort, wo Sollwerte, Stellgr\"o\ss en und St\"orungen einwirken und wie ein Regler darauf reagieren kann.

\medskip
Typische Aufgaben, f\"ur die Blockdiagramme genutzt werden:
\begin{itemize}
  \item Struktur und Signalrichtungen eines Regelkreises dokumentieren und diskutieren,
  \item aus verkn\"upften Bl\"ocken eine Gesamt\"ubertragungsfunktion herleiten (Verst\"arkung, Stabilit\"at, Genauigkeit),
  \item beurteilen, wie R\"uckf\"uhrung St\"orungen d\"ampft oder das Folgeverhalten verbessert.
\end{itemize}
Damit bilden Blockdiagramme die Br\"ucke zwischen physikalischem System, mathematischem Modell und Reglerentwurf.

\subsection{Bl\"ocke und \"Ubertragungsfunktionen}
Jeder Block steht f\"ur eine \"Ubertragungsfunktion \(G(\p) = \dfrac{Y(\p)}{U(\p)}\) zwischen Eingang \(u(t)\) und Ausgang \(y(t)\). Entlang eines Signalwegs werden die \"Ubertragungsfunktionen multipliziert, parallele Wege addieren sich --- genau so, wie es die algebraischen Gleichungen im Laplace-Bereich vorgeben.
\begin{center}
  \begin{tikzpicture}[node distance=12mm]
    \node (u) {$U(\p)$};
    \node[block, right=of u] (g) {$G(\p)$};
    \node[right=of g] (y) {$Y(\p)$};
    \draw[line] (u) -- (g) -- (y);
  \end{tikzpicture}
\end{center}
\[
  Y(\p) = G(\p)\, U(\p)
\]
Die grafische Darstellung codiert also direkt die Gleichungen: jeder Summierknoten entspricht einer Addition/Subtraktion, jeder Abzweig einer Signalweitergabe, jeder Block einer Multiplikation mit seiner \"Ubertragungsfunktion.

\subsection{Offener vs.\ geschlossener Regelkreis}
Ein \emph{offener} Kreis besitzt keine R\"uckf\"uhrung; die Regelgr\"o\ss e \(x\) beeinflusst den Eingang nicht. Ein \emph{geschlossener} Kreis f\"uhrt das Ausgangssignal (oft nach Messung \(H(\p)\)) zur\"uck, vergleicht es mit dem Sollwert \(w\) und erzeugt daraus die Stellgr\"o\ss e \(y\) f\"ur die Strecke.
\begin{center}
  \begin{tikzpicture}[node distance=12mm]
    \node (w) {$w$};
    \node[block, right=of w] (c) {$G_R$};
    \node[block, right=of c] (p) {$G_S$};
    \node[right=of p] (x) {$x$};
    \draw[line] (w) -- (c) -- (p) -- (x);
    \node at ($(c)!0.5!(p)+(0,7mm)$) {\small $y$};
    \node at ($(p)!0.5!(x)+(0,7mm)$) {\small offener Kreis};
  \end{tikzpicture}
  \qquad
  \begin{tikzpicture}[node distance=12mm]
    \node (w) {$w$};
    \node[sum, right=of w] (s) {};
    \node[block, right=of s] (c) {$G_R$};
    \node[block, right=of c] (p) {$G_S$};
    \node[right=of p] (x) {$x$};
    \node[block, below=10mm of p] (h) {$H$};
    \draw[line] (w) -- (s) -- (c) -- (p) -- (x);
    \draw[line] (x) |- (h);
    \draw[line] (h) -| (s);
    \node at ($(s)+(-3.5mm,2.2mm)$) {$+$};
    \node at ($(s)+(-3.5mm,-2.2mm)$) {$-$};
    \node at ($(c)!0.5!(p)+(0,7mm)$) {\small $y$};
    \node at ($(p)!0.5!(x)+(0,7mm)$) {\small geschlossener Kreis};
  \end{tikzpicture}
\end{center}
Vom offenen zum geschlossenen Kreis gelangt man also durch Hinzuf\"ugen der Summierstelle (Sollwertvergleich), eines Messpfads \(H(\p)\) und der R\"uckf\"uhrleitung. Die geschlossene \"Ubertragungsfunktion wird damit zu
\[
  T(\p) = \frac{X(\p)}{W(\p)} = \frac{G_R(\p) G_S(\p)}{1 + G_R(\p) G_S(\p) H(\p)},
\]
w\"ahrend der offene Weg lediglich \(G_R(\p) G_S(\p)\) bildet. Gerade in der Regelungstechnik ist die Wahl und Umformung des Blockdiagramms zentral, um St\"orunterdr\"uckung, Dynamik und Stabilit\"at zu beurteilen und Reglerparameter gezielt anzupassen.

\subsection{Regeln zur Umformung (mit Formeln und Bildern)}
\textbf{Voraussetzungen:} Die Umformregeln gelten f\"ur \emph{lineare zeitinvariante} (LZI) Systeme. Im Folgenden sind alle Signale im Laplace-Bereich angegeben (z.\,B.\ \(X(\p)\)); f\"ur die Umformung selbst gen\"ugt die Darstellung mit \"Ubertragungsfunktionen.

\medskip
\textbf{Konvention:} In vielen Skripten ist die \emph{negative R\"uckf\"uhrung} der Standardfall (Minuszeichen am Summierglied). Bei positiver R\"uckf\"uhrung \"andert sich das Vorzeichen im Nenner (siehe Regel~3).

\subsubsection{Regel 1: Reihenschaltung (Kaskade)}
Zwei hintereinander geschaltete Bl\"ocke lassen sich zu einem Block zusammenfassen:
\[
  G_{\mathrm{ges}}(\p) = G_2(\p)\,G_1(\p).
\]
\begin{center}
  \begin{tikzpicture}[node distance=12mm]
    \node (in) {$u$};
    \node[block, right=of in] (g1) {$G_1$};
    \node[block, right=of g1] (g2) {$G_2$};
    \node[right=of g2] (out) {$y$};
    \draw[line] (in) -- (g1) -- (g2) -- (out);
  \end{tikzpicture}
  \qquad$\Longleftrightarrow$\qquad
  \begin{tikzpicture}[node distance=12mm]
    \node (in) {$u$};
    \node[block, right=of in] (g) {$G_2 G_1$};
    \node[right=of g] (out) {$y$};
    \draw[line] (in) -- (g) -- (out);
  \end{tikzpicture}
\end{center}

\paragraph{Hinweis:} In einer reinen SISO-Kaskade darf man die Reihenfolge vertauschen \((G_2 G_1 = G_1 G_2)\). \emph{Nicht} vertauschen, wenn zwischen den Bl\"ocken Abzweige oder Summierstellen liegen.

\subsubsection{Regel 2: Parallelschaltung}
Parallele Wege mit gleicher Ein- und Ausgangsgr\"o\ss e addieren (oder subtrahieren) sich:
\[
  G_{\mathrm{ges}}(\p) = G_1(\p) \pm G_2(\p).
\]
\begin{center}
  \begin{tikzpicture}[node distance=12mm]
    \node (in) {$u$};
    \node[branch, right=of in] (b1) {};
    \node[block, above right=10mm and 12mm of b1] (g1) {$G_1$};
    \node[block, below right=10mm and 12mm of b1] (g2) {$G_2$};
    \node[sum, right=26mm of b1] (s) {};
    \node[right=of s] (out) {$y$};
    \draw[line] (in) -- (b1);
    \draw[line] (b1) |- (g1);
    \draw[line] (b1) |- (g2);
    \draw[line] (g1) -| (s);
    \draw[line] (g2) -| (s);
    \draw[line] (s) -- (out);
    \node at ($(s)+(-3.5mm,2.2mm)$) {$+$};
    \node at ($(s)+(-3.5mm,-2.2mm)$) {$\pm$};
  \end{tikzpicture}
  \qquad$\Longleftrightarrow$\qquad
  \begin{tikzpicture}[node distance=12mm]
    \node (in) {$u$};
    \node[block, right=of in] (g) {$G_1 \pm G_2$};
    \node[right=of g] (out) {$y$};
    \draw[line] (in) -- (g) -- (out);
  \end{tikzpicture}
\end{center}

\subsubsection{Regel 3: R\"uckf\"uhrung (Feedback)}
\paragraph{Negative R\"uckf\"uhrung (Standardfall):}
\[
  G_{\mathrm{cl}}(\p) = \frac{X(\p)}{W(\p)} = \frac{G(\p)}{1 + G(\p) H(\p)}.
\]
\begin{center}
  \begin{tikzpicture}[node distance=12mm]
    \node (w) {$w$};
    \node[sum, right=of w] (s) {};
    \node[block, right=of s] (g) {$G$};
    \node[right=of g] (x) {$x$};
    \node[block, below=10mm of g] (h) {$H$};
    \draw[line] (w) -- (s);
    \draw[line] (s) -- (g) -- (x);
    \draw[line] (x) |- (h);
    \draw[line] (h) -| (s);
    \node at ($(s)+(-3.5mm,2.2mm)$) {$+$};
    \node at ($(s)+(-3.5mm,-2.2mm)$) {$-$};
  \end{tikzpicture}
\end{center}

\paragraph{Positive R\"uckf\"uhrung:}
\[
  G_{\mathrm{cl}}(\p) = \frac{G(\p)}{1 - G(\p) H(\p)}.
\]

\paragraph{Kurzbeispiel (Ersatzblock).}
Gegeben sei eine Kaskade $G_1(\p)G_2(\p)$ mit Einheitsr\"uckf\"uhrung $H(\p)=1$.
Dann lautet die geschlossene \"Ubertragungsfunktion
\[
  G_{\mathrm{cl}}(\p)=\frac{G_1(\p)G_2(\p)}{1+G_1(\p)G_2(\p)}.
\]

\subsubsection{Regel 4: Summierstelle \"uber einen Block verschieben}
Diese Regel ist eine h\"aufige Fehlerquelle: \textbf{Beim Verschieben muss das Signal an jedem Knoten identisch bleiben.}

\paragraph{(a) Summierstelle \(\rightarrow\) nach \(\boldsymbol{G}\) verschieben:}
Aus \(y = G(u \pm v)\) folgt \(y = G u \pm G v\). Wird die Summierung \emph{hinter} \(G\) platziert, muss der zweite Zweig ebenfalls mit \(G\) gewichtet werden.
\begin{center}
  \begin{tikzpicture}[node distance=12mm]
    \node (u) {$u$};
    \node[sum, right=of u] (s) {};
    \node[block, right=of s] (g) {$G$};
    \node[right=of g] (y) {$y$};
    \node (v) [below=10mm of s] {$v$};
    \draw[line] (u) -- (s) -- (g) -- (y);
    \draw[line] (v) -- (s);
    \node at ($(s)+(-3.5mm,2.2mm)$) {$+$};
    \node at ($(s)+(-3.5mm,-2.2mm)$) {$\pm$};
  \end{tikzpicture}
  \qquad$\Longleftrightarrow$\qquad
  \begin{tikzpicture}[node distance=12mm]
    \node (u) {$u$};
    \node[block, right=of u] (g1) {$G$};
    \node[sum, right=of g1] (s) {};
    \node[right=of s] (y) {$y$};
    \node (v) [below=10mm of s] {$v$};
    \node[block, left=of v] (g2) {$G$};
    \draw[line] (u) -- (g1) -- (s) -- (y);
    \draw[line] (v) -- (g2) -- (s);
    \node at ($(s)+(-3.5mm,2.2mm)$) {$+$};
    \node at ($(s)+(-3.5mm,-2.2mm)$) {$\pm$};
  \end{tikzpicture}
\end{center}

\paragraph{(b) Summierstelle \(\leftarrow\) vor \(\boldsymbol{G}\) verschieben:}
Umgekehrt gilt: Aus \(y = G u \pm v\) wird \(y = G \bigl(u \pm v / G\bigr)\). Beim Verschieben \emph{vor} \(G\) muss der Zweig, der \(G\) zuvor \emph{nicht} durchlief, mit \(1/G\) skaliert werden.

\subsubsection{Regel 5: Abzweigstelle (Messabgriff) \"uber einen Block verschieben}
Auch hier gilt: das abgezweigte Signal muss gleich bleiben.

\paragraph{(a) Abzweig \(\rightarrow\) nach \(\boldsymbol{G}\) verschieben:}
Wird hinter \(G\) abgegriffen, ist das Signal um \(G\) gr\"o\ss er; daher muss im Abzweig ein Faktor \(1/G\) eingef\"ugt werden, um das urspr\"ungliche Signal beizubehalten.
\begin{center}
  \begin{tikzpicture}[node distance=12mm]
    \node (u) {$u$};
    \node[branch, right=of u] (b) {};
    \node[block, right=of b] (g) {$G$};
    \node[right=of g] (y) {$y$};
    \node[below=10mm of b] (tap) {$u$};
    \draw[line] (u) -- (b) -- (g) -- (y);
    \draw[line] (b) -- (tap);
  \end{tikzpicture}
  \qquad$\Longleftrightarrow$\qquad
  \begin{tikzpicture}[node distance=12mm]
    \node (u) {$u$};
    \node[block, right=of u] (g) {$G$};
    \node[branch, right=of g] (b) {};
    \node[right=of b] (y) {$y$};
    \node[below=10mm of b] (tap) {$u$};
    \node[block, left=of tap] (inv) {$1/G$};
    \draw[line] (u) -- (g) -- (b) -- (y);
    \draw[line] (b) -- (inv) -- (tap);
  \end{tikzpicture}
\end{center}

\paragraph{(b) Abzweig \(\leftarrow\) vor \(\boldsymbol{G}\) verschieben:}
Analog: Beim Verschieben eines Abgriffs \emph{vor} \(G\) wird im Abzweig ein Faktor \(G\) ben\"otigt.

\subsubsection{Regel 6: Innere Schleifen zuerst reduzieren}
Bei verschachtelten R\"uckf\"uhrungen ist es oft am einfachsten,
\begin{itemize}
  \item zun\"achst \emph{innere} Feedback-Schleifen mit Regel~3 zu einem Ersatzblock zusammenzufassen,
  \item danach Serien- und Parallelbl\"ocke (Regeln~1--2) zu reduzieren,
  \item und zuletzt \"au\ss ere Schleifen zu schlie\ss en.
\end{itemize}

\paragraph{Praxis-Tipp:} Zeichne bei jeder Umformung kurz die \emph{Signalgleichungen} an den Knoten (z.\,B.\ \(y = G u\), \(e = w - x\)). Wenn die Gleichungen identisch bleiben, ist die Umformung korrekt.

% kapitel/einschleifiger-regelkreis.tex
\section{Einschleifiger Regelkreis: Standardformen}
\subsection{Struktur und Grundgleichungen}
Wir betrachten die Standardstruktur mit F\"uhrungsgr\"o\ss e $w$, Regelabweichung $e$, Stellgr\"o\ss e $y$,
St\"orgr\"o\ss e $z$ (am Streckeneingang) und Regelgr\"o\ss e $x$:
\[
  e=w-x,\qquad y=G_R(p)\,e,\qquad x=G_S(p)\,(y+z).
\]
Die Kreisverst\"arkung (aufgeschnittener Kreis) ist
\[
  G_0(p)=G_R(p)\,G_S(p).
\]
Die Grafik fasst diese Standardstruktur zusammen (Einheitsmessung, St\"orung $z$ vor der Strecke):
\begin{center}
  \begin{tikzpicture}[node distance=12mm]
    \node (w) {$w$};
    \node[sum, right=of w] (s1) {};
    \node[block, right=of s1] (gr) {$G_R$};
    \node[sum, right=of gr] (s2) {};
    \node[block, right=of s2] (gs) {$G_S$};
    \node[right=of gs] (x) {$x$};
    \node[below=10mm of s2] (z) {$z$};
    \node[branch, below=10mm of x] (bfb) {};
    \draw[line] (w) -- (s1) -- node[above,pos=0.5] {$e$} (gr) -- node[above,pos=0.5] {$y$} (s2) -- (gs) -- (x);
    \draw[line] (z) -- (s2);
    \draw[line] (x) |- (bfb) -| (s1.south);
    \node at ($(s1)+(-3.5mm,2.2mm)$) {$+$};
    \node at ($(s1)+(-3.5mm,-2.2mm)$) {$-$};
    \node at ($(s2)+(-3.5mm,2.2mm)$) {$+$};
    \node at ($(s2)+(-3.5mm,-2.2mm)$) {$+$};
  \end{tikzpicture}
\end{center}
Dabei erzeugt der Regler $G_R$ aus der Regelabweichung $e$ die Stellgr\"o\ss e $y$, die Strecke $G_S$ setzt $y$ (plus St\"orung $z$) in die Regelgr\"o\ss e $x$ um, und die R\"uckf\"uhrung sorgt daf\"ur, dass $e$ im Idealfall klein wird.

\subsection{F\"uhrungs- und St\"or\"ubertragungsfunktion}
\paragraph{F\"uhrungs\"ubertragung ($z=0$).}
\[
  G_w(p)=\frac{X(p)}{W(p)}=\frac{G_0(p)}{1+G_0(p)}.
\]

\paragraph{St\"or\"ubertragung ($w=0$).}
\[
  G_z(p)=\frac{X(p)}{Z(p)}=\frac{G_S(p)}{1+G_0(p)}.
\]

\subsection{Statisches Verhalten, Regelfaktor}
Das statische Verhalten ergibt sich aus $p\to 0$ (sofern Grenzwerte existieren):
\[
  V_0:=G_0(0)\quad\text{(statische Kreisverst\"arkung)}.
\]
Der Regelfaktor (Unterdr\"uckung von St\"orungen bzw.\ Abweichungen im statischen Fall) lautet
\[
  R=\frac{1}{1+V_0}.
\]
Damit gilt: gro\ss e Kreisverst\"arkung $V_0\gg 1 \Rightarrow$ kleines $R$ (gute Unterdr\"uckung).

\paragraph{Beispiel (Regelfaktor).}
Sei $G_R(p)=K_{PR}$ und $G_S(p)=\dfrac{K_S}{1+Tp}$. Dann ist
$V_0=K_{PR}K_S$ und damit
\[
  R=\frac{1}{1+K_{PR}K_S}.
\]

\subsection{P-, I- und PI-Regler (Grundideen)}
\begin{center}
  \begin{tabular}{@{}lll@{}}
    \toprule
    Regler & \"Ubertragungsfunktion $G_R(p)$ & typische Wirkung \\
    \midrule
    P  & $K_{PR}$ & schnell, aber i.\,A.\ bleibende Abweichung \\
    I  & $\displaystyle \frac{K_{IR}}{p}$ & keine bleibende Abweichung, aber h\"ohere Ordnung/Schwingneigung \\
    PI & $\displaystyle K_{PR}\!\left(1+\frac{1}{T_N p}\right)$ & kombiniert Vorteile von P und I \\
    \bottomrule
  \end{tabular}
\end{center}
Parameterbezug beim PI-Regler:
\[
  K_{IR}=\frac{K_{PR}}{T_N}.
\]

% kapitel/stabilitaet-linearer-regelkreise.tex
\section{Stabilit\"at linearer Regelkreise}
\subsection{Charakteristische Gleichung}
Ein LZI-System ist (asymptotisch) stabil, wenn alle Pole strikt in der linken Halbebene liegen
($\Ree(p_i)<0$). F\"ur den einschleifigen Regelkreis folgt die charakteristische Gleichung aus dem Nenner:
\[
  1+G_0(p)=0.
\]

\subsection{Nyquist-Idee (qualitativ)}
F\"ur sinusf\"ormige Anregung sind im eingeschwungenen Zustand alle Signale sinusf\"ormig
(mit anderer Amplitude und Phasenlage). Mit $\omega_k$ bezeichnen wir eine beliebige abgetastete
Frequenz. Wird bei einer Frequenz $\omega_k$ eine Phasendrehung
von $\varphi(\omega_k)=-180^\circ$ erreicht, wirkt die R\"uckkopplung wie positive R\"uckkopplung.
Dann entscheidet der Betrag:
\[
  \lvert G_0(\jj\omega_k)\rvert
  \begin{cases}
    <1 & \Rightarrow \text{Schwingungen klingen ab (stabil)},\\
    =1 & \Rightarrow \text{Grenzstabilit\"at (Dauerschwingung)},\\
    >1 & \Rightarrow \text{Schwingungen wachsen an (instabil)}.
  \end{cases}
\]
F\"ur Systeme ohne Pole in der rechten Halbebene kann man in vielen Anwendungen die vereinfachte
Schnittbedingung nutzen: An Frequenzen mit $\Imm(G_0(\jj\omega))=0$ muss
$\Ree(G_0(\jj\omega))>-1$ gelten.

\paragraph{Warum die Bedingung $\Ree(G_0(\jj\omega))>-1$?}
Auf der reellen Achse ($\Imm=0$) hat $G_0(\jj\omega)$ die Phasenlage $\varphi=0^\circ$ oder
$-180^\circ$. Erreicht die Kurve dort $\Ree\le -1$, dann liegt der kritische Punkt $-1+\jj0$ auf
oder links des aktuellen Kurvenpunktes. Das entspricht Betrag $\lvert G_0\rvert\ge 1$ bei
$\varphi=-180^\circ$ und damit genau der Grenze (bzw.\ dem \"Ubertritt) von stabil zu instabil:
$1+G_0(\jj\omega)=0$ ist erf\"ullt (der Punkt $-1+\jj0$ wird getroffen), oder $-1$ wird umschlungen.
Liegt der Realteil dagegen gr\"o\ss er als $-1$, befindet sich die Ortskurve auf der ``stabilen''
Seite des kritischen Punkts, d.\,h.\ bei $-180^\circ$ ist der Betrag kleiner als 1 und der
Nyquist-Test bleibt erf\"ullt.

\begin{center}
  \begin{tikzpicture}[scale=1.0]
    % axes
    \draw[->,thin] (-2.5,0) -- (1.5,0) node[right] {$\Ree$};
    \draw[->,thin] (0,-1.4) -- (0,1.4) node[above] {$\Imm$};
    % critical point -1
    \filldraw (-1,0) circle (1.2pt);
    \node[below left] at (-1,0) {$-1$};
    % stable crossing (Re>-1)
    \draw[line width=0.9pt,blue] plot[smooth] coordinates {(-0.6,-1.0) (-0.7,-0.4) (-0.8,0) (-0.7,0.5) (-0.4,1.0)};
    \node[blue,align=left] at (-0.15,1.05) {\footnotesize stabiler Schnitt:\\[-1pt] $|G_0|<1$ bei $-180^\circ$};
    % unstable crossing (Re<=-1)
    \draw[line width=0.9pt,red,dashed] plot[smooth] coordinates {(-1.4,-1.0) (-1.2,-0.4) (-1.1,0) (-1.2,0.5) (-1.5,1.0)};
    \node[red,align=left] at (-1.95,1.05) {\footnotesize instabiler Schnitt:\\[-1pt] $|G_0|\ge 1$ bei $-180^\circ$};
    % arrows showing direction (increasing omega)
    \draw[blue,-Latex] (-0.74,-0.1) -- (-0.78,0.15);
    \draw[red,-Latex] (-1.16,-0.1) -- (-1.20,0.15);
  \end{tikzpicture}
\end{center}
Die blaue Kurve schneidet die reelle Achse rechts von $-1$ ($|G_0|<1$ bei $-180^\circ$) und umschlingt
$-1$ nicht: der geschlossene Kreis bleibt stabil. Die rote, gestrichelte Kurve kreuzt links von $-1$
($|G_0|\ge 1$ bei $-180^\circ$); damit wird $-1$ getroffen oder umschlungen und mindestens ein
geschlossener Pol wandert in die rechte Halbebene (Instabilit\"at).

\subsection{Nyquist-Diagramm und Nyquist-Kriterium}
\paragraph{Nyquist-Diagramm (Nyquist-Ortskurve).}
F\"ur die Kreisverst\"arkung (aufgeschnittener Regelkreis)
\[
  G_0(p)=G_R(p)\,G_S(p)
\]
hei\ss t \emph{Nyquist-Diagramm} die Ortskurve der komplexen Werte $G_0(\jj\omega)$
in der $\Ree$--$\Imm$-Ebene, wenn $\omega$ von $0$ bis $\infty$ l\"auft.
F\"ur Systeme mit reellen Koeffizienten ergibt sich die vollst\"andige Nyquist-Kurve
durch Spiegelung an der reellen Achse.

\paragraph{Stabilit\"atsfrage \"uber $1+G_0(p)=0$.}
F\"ur den einschleifigen Regelkreis mit Einheitsr\"uckf\"uhrung ist die charakteristische Gleichung
\[
  1+G_0(p)=0.
\]
Die Nullstellen von $1+G_0(p)$ sind die Pole des geschlossenen Kreises
(Stabilit\"at $\Leftrightarrow$ alle Pole in der linken Halbebene).

\paragraph{Nyquist-Kriterium (Vorzeichenkonvention).}
Sei $P$ die Anzahl der Pole von $G_0(p)$ in der rechten Halbebene und $Z$ die Anzahl der
Nullstellen von $1+G_0(p)$ in der rechten Halbebene (instabile geschlossene Pole).
Weiter sei $N$ die Anzahl der \emph{Uhrzeigersinn}-Umschlingungen des Punktes $-1+\jj0$
durch die vollst\"andige Nyquist-Kurve von $G_0(\jj\omega)$. Dann gilt
\[
  N=Z-P
  \qquad\Leftrightarrow\qquad
  Z=N+P.
\]
Damit folgt unmittelbar:
\[
  \text{geschlossener Kreis stabil} \;\;\Leftrightarrow\;\; Z=0 \;\;\Leftrightarrow\;\; N=-P.
\]
Insbesondere: Ist der aufgeschnittene Kreis bereits stabil ($P=0$), dann ist der
geschlossene Kreis genau dann stabil, wenn die Nyquist-Kurve $-1$ \emph{nicht}
im Uhrzeigersinn umschlingt ($N=0$).
Ein formaler Beweis des Nyquist-Kriteriums (über die Argument Principle) wird hier nicht geführt;
wir nutzen das Ergebnis als praktisches Stabilitätswerkzeug.

\paragraph{Beispiel.}
Abbildung~\ref{fig:nyquist_example} zeigt
$G_0(p)=\dfrac{K}{(1+p)(1+0{,}2p)(1+0{,}05p)}$.
Bei $K\approx31{,}5$ verl\"auft die Ortskurve durch $-1$ (Grenzstabilit\"at);
f\"ur gr\"o\ss eres $K$ entstehen instabile Pole im geschlossenen Kreis.
\begin{itemize}[itemsep=2pt]
  \item $K=20$ (durchgezogene Linie): Ortskurve bleibt rechts von $-1$ und umschlingt $-1$ nicht
    ($N=0$, $P=0 \Rightarrow Z=0$) \(\Rightarrow\) stabil.
  \item $K\approx31{,}5$ (gestrichelt): Ortskurve l\"auft durch $-1$ \(\Rightarrow\) Grenzfall, der
    geschlossene Kreis besitzt Pole auf der imagin\"aren Achse (Dauerschwingung).
  \item $K=40$ (punktiert): Ortskurve umschlingt $-1$ im Uhrzeigersinn einmal
    ($N=1$, $P=0 \Rightarrow Z=1$) \(\Rightarrow\) ein instabiler Pol im geschlossenen Kreis.
\end{itemize}

% Neue Abbildung 7 (Zoom) aus Abbildung7_neu_Paket_ZOOM
\begin{figure}[tb]
  \centering
  \includegraphics[width=0.8\linewidth]{Abbildung7_neu_Paket_ZOOM/Abbildung7_neu_latex.pdf}
  \caption{Nyquist-Ortskurven (Zoom um $(-1,0)$) der Kreisverstärkung
  $G_0(p)=\dfrac{K}{(1+p)(1+0{,}2p)(1+0{,}05p)}$ (nur $\omega>0$; vollständige Kurve durch Spiegelung).
  Stabil: $K=20$ bleibt rechts und berührt $(-1,0)$ nicht.
  Grenzstabil: $K\approx 31{,}50$ trifft $(-1,0)$.
  Instabil: $K=40$ umschlingt $(-1,0)$ im Uhrzeigersinn.}
  \label{fig:nyquist_example}
\end{figure}


\paragraph{Praktische Gr\"o\ss en: Verst\"arkungs- und Phasenreserve.}
Aus dem Nyquist- bzw.\ Bode-Diagramm lassen sich Stabilit\"atsreserven ablesen:
\begin{itemize}[itemsep=2pt]
  \item \textbf{Durchtrittsfrequenz} $\omega_c$: $|G_0(\jj\omega_c)|=1$.
    \textbf{Phasenreserve} $\varphi_m = 180^\circ + \arg(G_0(\jj\omega_c))$.
  \item \textbf{Phasendurchtritt} $\omega_\pi$: $\arg(G_0(\jj\omega_\pi))=-180^\circ$ (falls vorhanden).
    \textbf{Verst\"arkungsreserve} $G_m = 1/|G_0(\jj\omega_\pi)|$
    (in dB: $20\log_{10}G_m$).
\end{itemize}
Gro\ss e Reserven bedeuten typischerweise: Die Nyquist-Kurve bleibt deutlich vom kritischen
Punkt $-1$ entfernt.

\subsection{Hurwitz-Kriterium (Polynomkriterium)}
Bringt man die charakteristische Gleichung in die Polynomform
\[
  a_0+a_1p+a_2p^2+\cdots+a_np^n=0,
\]
dann ist eine notwendige Bedingung f\"ur Stabilit\"at $a_i>0$ f\"ur alle $i$.
Das Hurwitz-Kriterium liefert eine notwendige und hinreichende Bedingung:
Alle Hurwitz-Determinanten m\"ussen positiv sein.

\paragraph{Hurwitz-Matrix (Schema).}
\[
  H=
  \begin{pmatrix}
    a_1 & a_3 & a_5 & a_7 & \cdots\\
    a_0 & a_2 & a_4 & a_6 & \cdots\\
    0   & a_1 & a_3 & a_5 & \cdots\\
    0   & a_0 & a_2 & a_4 & \cdots\\
    \vdots & \vdots & \vdots & \vdots & \ddots
  \end{pmatrix}
\]
Die Hurwitz-Determinanten $H_k$ sind die Determinanten der linken oberen $k\times k$-Untermatrix, z.\,B.:
\[
  H_1 = \begin{vmatrix} a_1 \end{vmatrix} = a_1,\qquad
  H_2 = \begin{vmatrix} a_1 & a_3 \\[4pt] a_0 & a_2 \end{vmatrix} = a_1 a_2 - a_0 a_3,
  \qquad
  H_3 = \begin{vmatrix}
    a_1 & a_3 & a_5 \\
    a_0 & a_2 & a_4 \\
    0   & a_1 & a_3
  \end{vmatrix}.
\]
Im Allgemeinen gilt: $H_k$ ist die Determinante der $k\times k$-Untermatrix in der linken oberen Ecke von $H$.
Stabilit\"at $\Leftrightarrow H_k>0$ f\"ur alle $k=1,\ldots,n$.

\paragraph{Praktische Kurzform ($n=3$).}
\[
  a_0>0,\ a_1>0,\ a_2>0,\ a_3>0
  \quad\text{und}\quad
  a_1a_2>a_0a_3.
\]
Diese Kurzform ist genau die Auswertung von $H_1>0$, $H_2>0$ und $H_3>0$ f\"ur ein Polynom 3.\ Ordnung:
Positive Koeffizienten stellen $H_1$ sicher, $a_1a_2>a_0a_3$ entspricht $H_2>0$, und bei drei Polynomen
gen\"ugen damit alle Bedingungen des Hurwitz-Kriteriums.

\paragraph{Beispiel ($n=2$).}
F\"ur $p^2+2p+5=0$ gilt $a_2=1$, $a_1=2$, $a_0=5$.
Alle Koeffizienten sind positiv, daher ist das System stabil.

\subsection{Beispiel: PT2-Strecke mit PI-Regler (Stabilit\"atsgebiet)}
Sei
\[
  G_S(p)=\frac{K_S}{1+2DTp+T^2p^2},
  \qquad
  G_R(p)=K_{PR}+\frac{K_{IR}}{p}.
\]
Dann ist
\[
  G_0(p)=G_R(p)G_S(p)=\left(K_{PR}+\frac{K_{IR}}{p}\right)\frac{K_S}{1+2DTp+T^2p^2}.
\]
Charakteristische Gleichung:
\[
  1+G_0(p)=0
  \ \Leftrightarrow\
  p(1+2DTp+T^2p^2)+K_S K_{PR}\,p + K_S K_{IR}=0.
\]
Damit ergibt sich das Stabilit\"atspolynom 3.\ Ordnung:
\[
  a_3p^3+a_2p^2+a_1p+a_0=0
\]
mit
\[
  a_3=T^2,\quad a_2=2DT,\quad a_1=1+K_S K_{PR},\quad a_0=K_S K_{IR}.
\]
Hurwitz ($n=3$) liefert:
\[
  K_{IR}>0,\qquad
  1+K_S K_{PR}>0\ \Leftrightarrow\ K_{PR}>-\frac{1}{K_S},
\]
\[
  (1+K_S K_{PR})\,2DT > (K_S K_{IR})\,T^2
  \ \Leftrightarrow\
  K_{IR}<\frac{2D}{K_S T}\,(1+K_S K_{PR}).
\]
Diese Ungleichungen sind exakt die Bedingungen $H_1>0$, $H_2>0$, $H_3>0$ des Hurwitz-Kriteriums
f\"ur das Polynom 3.\ Ordnung: $H_1>0$ fordert $a_1>0$ ($\Rightarrow K_{PR}>-1/K_S$), $a_0>0$ liefert
$K_{IR}>0$, und $H_2>0$ ergibt die Schranke $K_{IR}<\tfrac{2D}{K_S T}(1+K_S K_{PR})$.
Damit l\"asst sich das Stabilit\"atsgebiet im $(K_{PR},K_{IR})$-Diagramm grafisch darstellen.

\paragraph{Beispielhafte Parameterwahl.}
F\"ur $K_S=1$, $D=0{,}5$, $T=1$ wird daraus:
\[
  K_{IR}>0,\qquad K_{PR}>-1,\qquad K_{IR}<1+K_{PR}.
\]
Damit ist das Stabilit\"atsgebiet ein einfaches Dreieck (Abbildung~\ref{fig:stability_region_pi_pt2}).

% Stability region for PT2 plant with PI controller (example parameters)
\begin{figure}[ht]
  \centering
  \begin{tikzpicture}
    \begin{axis}[
      width=0.70\textwidth, height=0.45\textwidth,
      grid=both, grid style={line width=.1pt, draw=gray!20},
      major grid style={line width=.2pt, draw=gray!35},
      xlabel={$K_{PR}$}, ylabel={$K_{IR}$},
      xmin=-1.2, xmax=5.2,
      ymin=-0.2, ymax=6.2,
      axis lines=left,
      tick label style={font=\small},
      label style={font=\small},
    ]
      % Filled stability region polygon
      \addplot[fill=gray!15, draw=none] table[x=KPR,y=KIR,col sep=comma] {data/stab_region_polygon_PI_PT2_KS1_D0p5_T1.csv} \closedcycle;
      % Boundary lines
      \addplot[thick] table[x=KPR,y=KIR_max,col sep=comma] {data/stab_region_PI_PT2_KS1_D0p5_T1.csv};
      \addplot[thick] coordinates {(-1,0) (5,0)};
      \addplot[dashed] coordinates {(-1,0) (-1,6)};
      \node[font=\small, anchor=west] at (axis cs:-0.95,5.55) {$K_S=1,\;D=0{,}5,\;T=1$};
      \node[font=\small, anchor=west] at (axis cs:-0.95,4.95) {$0<K_{IR}<1+K_{PR},\;\;K_{PR}>-1$};
    \end{axis}
  \end{tikzpicture}
  \caption{Beispielhaftes Stabilit\"atsgebiet f\"ur eine PT2-Strecke mit PI-Regler (Parameterwahl $K_S=1$, $D=0{,}5$, $T=1$). Das schattierte Gebiet erf\"ullt die Hurwitz-Bedingungen $H_1,H_2,H_3$ aus Abschnitt~11.4.}
  \label{fig:stability_region_pi_pt2}
\end{figure}


% kapitel/reglerstrukturen-realisierung.tex
\section{Reglerstrukturen (PD, PID) und Realisierungshinweise}
\subsection{PD- und PID-Formen}
\paragraph{PD-Regler.}
\[
  G_R(p)=K_{PR}+K_{DR}p
  =K_{PR}\left(1+T_V p\right),
  \qquad
  T_V=\frac{K_{DR}}{K_{PR}}.
\]

\paragraph{Realisierbarer PD (mit Realisierungspol).}
Ein idealer Differenzierer verst\"arkt hohe Frequenzen unbegrenzt; daher wird h\"aufig ein Pol eingef\"uhrt, z.\,B.
\[
  G_R(p)=K_{PR}\left(1+\frac{T_V p}{1+T_R p}\right),
\]
wobei $T_R$ die Zeitkonstante des Realisierungspols ist. Der Realisierungspol wirkt als Tiefpass
im D-Anteil: f\"ur kleine Frequenzen gilt nahezu $T_V p$, f\"ur hohe Frequenzen wird die
Verst\"arkung begrenzt (Rauschunterdr\"uckung, physikalische Realisierbarkeit).

\paragraph{PID-Regler (ideal).}
\[
  G_R(p)=K_{PR}+\frac{K_{IR}}{p}+K_{DR}p
  =K_{PR}\left(1+\frac{1}{T_N p}+T_V p\right),
  \qquad
  T_N=\frac{K_{PR}}{K_{IR}}.
\]

\paragraph{PID-Regler (realisierbar).}
\[
  G_R(p)=K_{PR}\left(1+\frac{1}{T_N p}+\frac{T_V p}{1+T_R p}\right),
  \qquad T_R \ll T_V.
\]
Hier ist $T_N$ die Nachstellzeit (I-Anteil) und $T_R$ der Realisierungspol des D-Anteils.

\subsection{(Invertierender) OPV-Grundsatz}
F\"ur einen idealen invertierenden Operationsverst\"arker gilt im Laplace-Bereich:
\[
  \frac{U_a(p)}{U_e(p)}=-\frac{Z_2(p)}{Z_1(p)}.
\]
Durch geeignete Wahl von $Z_1,Z_2$ (Widerst\"ande/Kondensatoren) lassen sich P-, I-, D- und DT1-Verhalten realisieren.

\begin{center}
  \begin{tabular}{@{}llll@{}}
    \toprule
    Funktion & $Z_1(p)$ & $Z_2(p)$ & $G(p)=-Z_2/Z_1$ \\
    \midrule
    P   & $R_1$ & $R_0$ & $-\dfrac{R_0}{R_1}$ \\
    I   & $R_1$ & $\dfrac{1}{C_0p}$ & $-\dfrac{1}{R_1C_0p}$ \\
    D   & $\dfrac{1}{C_1p}$ & $R_0$ & $-R_0C_1p$ \\
    DT1 & $R_1+\dfrac{1}{C_1p}$ & $R_0$ & $-\dfrac{R_0C_1p}{1+R_1C_1p}$ \\
    \bottomrule
  \end{tabular}
\end{center}

\paragraph{Beispiel (I-Glied).}
Mit $R_1=10\,\text{k}\Omega$ und $C_0=1\,\mu\text{F}$ ergibt sich
$G(p)=-\dfrac{1}{R_1C_0p}=-\dfrac{1}{0{,}01\,p}$.


\appendix
% kapitel/anhang-formelsammlung.tex
\section{Mini-Formelsammlung (Laplace)}
\begin{center}
  \begin{tabular}{@{}ll@{}}
    \toprule
    Zeitfunktion $x(t)$ (f\"ur $t\ge 0$) & $X(p)=\mathcal{L}\{x(t)\}$ \\
    \midrule
    $\sigma(t)$ & $\dfrac{1}{p}$ \\
    $\delta(t)$ & $1$ \\
    $t\,\sigma(t)$ & $\dfrac{1}{p^2}$ \\
    $\mathrm{e}^{-at}$ $(a>0)$ & $\dfrac{1}{p+a}$ \\
    $\sin(\omega t)$ & $\dfrac{\omega}{p^2+\omega^2}$ \\
    $\cos(\omega t)$ & $\dfrac{p}{p^2+\omega^2}$ \\
    \bottomrule
  \end{tabular}
\end{center}

\section{Ausf\"uhrliche Aufgaben}
\label{app:aufgaben}
\begin{enumerate}[itemsep=4pt]
  \item \textbf{\"Ubertragungsfunktion.}
  Gegeben ist
  $T\dot x_a + x_a = K\,x_e + K_D\,\dot x_e$ mit Null-Anfangsbedingungen.
  Bestimmen Sie $G(p)$ und die Sprungantwort f\"ur $x_e(t)=\sigma(t)$.
  \item \textbf{Bode-Absch\"atzung.}
  F\"ur $G(p)=\dfrac{10}{(1+0{,}5p)(1+2p)}$ skizzieren Sie die asymptotischen
  Betragsgeraden und geben Sie die Eckfrequenzen an.
  \item \textbf{Regelkreis.}
  F\"ur $G_R(p)=K_{PR}\left(1+\dfrac{1}{T_N p}\right)$ und
  $G_S(p)=\dfrac{K_S}{1+Tp}$ bestimmen Sie $G_0(p)$, $G_w(p)$ und den
  statischen Regelfaktor $R$.
\end{enumerate}


\end{document}

