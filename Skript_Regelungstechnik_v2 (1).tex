\documentclass[11pt,a4paper]{article}

\usepackage[a4paper,margin=2.2cm]{geometry}
\usepackage[T1]{fontenc}
\usepackage[utf8]{inputenc}
\usepackage[ngerman]{babel}
\usepackage{microtype}
\usepackage{amsmath,amssymb,mathtools}
\usepackage{siunitx}
\usepackage{booktabs}
\usepackage{graphicx}
\usepackage{xcolor}
\usepackage{hyperref}
\usepackage{cleveref}

\usepackage{tikz}
\usetikzlibrary{arrows.meta,positioning,calc,shapes.geometric}

\hypersetup{colorlinks=true,linkcolor=blue,urlcolor=blue,citecolor=blue}

% ---------- Notation ----------
\newcommand{\Laplace}{\mathcal{L}}
\newcommand{\p}{p} % Laplace-Variable (in den Notizen meist p statt s)

% ---------- TikZ helpers ----------
\tikzset{
  block/.style={draw,rectangle,minimum height=9mm,minimum width=16mm,align=center},
  sum/.style={draw,circle,inner sep=0pt,minimum size=4.5mm},
  branch/.style={circle,fill,inner sep=0pt,minimum size=1.6mm},
  line/.style={-Latex,thick},
  thin/.style={-Latex,semithick}
}

\title{Skript Regelungstechnik (LaTeX-Version)}
\author{}
\date{\today}

\begin{document}
\maketitle
\tableofcontents

\section{Grundbegriffe}
\subsection{Lineare zeitinvariante Systeme (LZI)}
Ein System heißt \emph{linear}, wenn Superposition gilt, und \emph{zeitinvariant}, wenn eine Zeitverschiebung am Eingang dieselbe Zeitverschiebung am Ausgang bewirkt.

\subsection{Übertragungsfunktion}
Für LZI-Systeme mit Nullanfangsbedingungen definiert man die \emph{Übertragungsfunktion}
\[
G(\p)=\frac{X_a(\p)}{X_e(\p)}.
\]

\section{Laplace-Transformation (Kurzüberblick)}
\subsection{Definition und wichtige Paare}
\[
\Laplace\{x(t)\}(\p)=\int_0^{\infty} x(t)\,e^{-\p t}\,\mathrm{d}t\qquad (\Re(\p)>\sigma_0).
\]
Wichtige Transformationsregeln (Nullanfangsbedingungen):
\[
\Laplace\{\dot x(t)\}=\p X(\p),\qquad \Laplace\{\ddot x(t)\}=\p^2 X(\p).
\]
Sprung- und Rampenfunktion (Heaviside \(\sigma(t)\)):
\[
\Laplace\{\sigma(t)\}=\frac{1}{\p},\qquad \Laplace\{t\,\sigma(t)\}=\frac{1}{\p^2}.
\]

\section{Typische Übertragungsglieder}
\begin{center}
\begin{tabular}{@{}lll@{}}
\toprule
Glied & Zeitbereich (typisch) & Übertragungsfunktion \\
\midrule
P-Glied & \(x_a=K\,x_e\) & \(G(\p)=K\) \\
I-Glied & \(x_a=K\int x_e\,dt\) & \(G(\p)=\frac{K}{\p}\) \\
D-Glied & \(x_a=K\,\dot x_e\) & \(G(\p)=K\p\) \\
PT1 & \(T\dot x_a+x_a=Kx_e\) & \(G(\p)=\frac{K}{1+T\p}\) \\
PT2 & \(T^2\ddot x_a+2DT\dot x_a+x_a=Kx_e\) & \(G(\p)=\frac{K}{1+2DT\p+T^2\p^2}\) \\
\bottomrule
\end{tabular}
\end{center}

\section{Frequenzgang und Bode-Größen}
Frequenzgang erhält man durch Einsetzen \(\p=j\omega\): \(G(j\omega)\).
\[
|G(j\omega)|\quad \text{(Betrag)},\qquad \varphi(\omega)=\arg\,G(j\omega)\quad \text{(Phase)}.
\]
Bode-Betrag in Dezibel: \(L(\omega)=20\log_{10}|G(j\omega)|\,\text{dB}.\)

\section{Stabilität (sehr kurz)}
Ein LZI-System ist (asymptotisch) stabil, wenn alle Pole von \(G(\p)\) strikt in der linken Halbebene liegen (\(\Re(\p)<0\)).

\section{Regelkreise: Standardformeln}
\subsection{Grundstruktur}
Für den Standard-Regelkreis mit Regler \(G_R(\p)\), Strecke \(G_S(\p)\) und Messglied \(G_M(\p)\) (negative Rückführung) gilt:
\[
E(\p)=W(\p)-X(\p),\quad Y(\p)=G_R(\p)E(\p),\quad X(\p)=G_S(\p)\bigl(Y(\p)+Z(\p)\bigr),\quad X(\p)=G_M(\p)\,X_a(\p)
\]
(je nach Konvention sind \(X\) und \(X_a\) Mess- bzw. Regelgröße).

\subsection{Führungs- und Störübertragungsfunktion}
Für den üblichen Fall \(G_M(\p)=1\) ergibt sich
\[
\frac{X(\p)}{W(\p)}=\frac{G_R(\p)G_S(\p)}{1+G_R(\p)G_S(\p)}\qquad\text{(Führungsübertragung)}
\]
und
\[
\frac{X(\p)}{Z(\p)}=\frac{G_S(\p)}{1+G_R(\p)G_S(\p)}\qquad\text{(Störübertragung; Störung additiv am Streckeneingang)}.
\]

\section{Blockdiagramm-Umformung}
\subsection{8.1 Motivation}
Blockdiagramme lassen sich schrittweise vereinfachen, um z.\,B. die Gesamtübertragungsfunktion zu bestimmen oder die Wirkung von Störungen/Reglerparametern transparent zu machen.


\subsection{8.2 Regeln zur Umformung (mit Formeln und Bildern)}
\textbf{Voraussetzungen:} Die Umformregeln gelten für \emph{lineare zeitinvariante} (LZI) Systeme. 
Im Folgenden sind alle Signale im Laplace-Bereich angegeben (z.\,B.\ \(X(\p)\)); für die Umformung selbst genügt die Darstellung mit Übertragungsfunktionen.

\medskip
\textbf{Konvention:} In vielen Skripten ist die \emph{negative Rückführung} der Standardfall (Minuszeichen am Summierglied).
Bei positiver Rückführung ändert sich das Vorzeichen im Nenner (siehe Regel~3).

\subsubsection{Regel 1: Reihenschaltung (Kaskade)}
Zwei hintereinander geschaltete Blöcke lassen sich zu einem Block zusammenfassen:
\[
G_{\mathrm{ges}}(\p)=G_2(\p)\,G_1(\p).
\]
\begin{center}
\begin{tikzpicture}[node distance=12mm]
  \node (in) {$u$};
  \node[block,right=of in] (g1) {$G_1$};
  \node[block,right=of g1] (g2) {$G_2$};
  \node[right=of g2] (out) {$y$};
  \draw[line] (in) -- (g1) -- (g2) -- (out);
\end{tikzpicture}
\qquad$\Longleftrightarrow$\qquad
\begin{tikzpicture}[node distance=12mm]
  \node (in) {$u$};
  \node[block,right=of in] (g) {$G_2G_1$};
  \node[right=of g] (out) {$y$};
  \draw[line] (in) -- (g) -- (out);
\end{tikzpicture}
\end{center}

\paragraph{Hinweis:} In einer reinen SISO-Kaskade darf man die Reihenfolge vertauschen (\(G_2G_1=G_1G_2\)). 
\emph{Nicht} vertauschen, wenn zwischen den Blöcken Abzweige/Summierstellen liegen.

\subsubsection{Regel 2: Parallelschaltung}
Parallele Wege mit gleicher Ein- und Ausgangsgröße addieren (oder subtrahieren) sich:
\[
G_{\mathrm{ges}}(\p)=G_1(\p)\pm G_2(\p).
\]
\begin{center}
\begin{tikzpicture}[node distance=12mm]
  \node (in) {$u$};
  \node[branch,right=of in] (b1) {};
  \node[block,above right=10mm and 12mm of b1] (g1) {$G_1$};
  \node[block,below right=10mm and 12mm of b1] (g2) {$G_2$};
  \node[sum,right=26mm of b1] (s) {};
  \node[right=of s] (out) {$y$};

  \draw[line] (in) -- (b1);
  \draw[line] (b1) |- (g1);
  \draw[line] (b1) |- (g2);
  \draw[line] (g1) -| (s);
  \draw[line] (g2) -| (s);
  \draw[line] (s) -- (out);

  \node at ($(s)+(-3.5mm,2.2mm)$) {$+$};
  \node at ($(s)+(-3.5mm,-2.2mm)$) {$\pm$};
\end{tikzpicture}
\qquad$\Longleftrightarrow$\qquad
\begin{tikzpicture}[node distance=12mm]
  \node (in) {$u$};
  \node[block,right=of in] (g) {$G_1\pm G_2$};
  \node[right=of g] (out) {$y$};
  \draw[line] (in) -- (g) -- (out);
\end{tikzpicture}
\end{center}

\subsubsection{Regel 3: Rückführung (Feedback)}
\paragraph{Negative Rückführung (Standardfall):}
\[
G_{\mathrm{cl}}(\p)=\frac{X(\p)}{W(\p)}=\frac{G(\p)}{1+G(\p)H(\p)}.
\]
\begin{center}
\begin{tikzpicture}[node distance=12mm]
  \node (w) {$w$};
  \node[sum, right=of w] (s) {};
  \node[block, right=of s] (g) {$G$};
  \node[right=of g] (x) {$x$};
  \node[block, below=10mm of g] (h) {$H$};

  \draw[line] (w) -- (s);
  \draw[line] (s) -- (g) -- (x);
  \draw[line] (x) |- (h);
  \draw[line] (h) -| (s);

  \node at ($(s)+(-3.5mm,2.2mm)$) {$+$};
  \node at ($(s)+(-3.5mm,-2.2mm)$) {$-$};
\end{tikzpicture}
\end{center}

\paragraph{Positive Rückführung:}
\[
G_{\mathrm{cl}}(\p)=\frac{G(\p)}{1-G(\p)H(\p)}.
\]

\subsubsection{Regel 4: Summierstelle über einen Block verschieben}
Diese Regel ist eine häufige Fehlerquelle: \textbf{Beim Verschieben muss das Signal an jedem Knoten identisch bleiben.}

\paragraph{(a) Summierstelle \(\rightarrow\) nach \(\boldsymbol{G}\) verschieben:}
Aus \(y=G(u\pm v)\) folgt \(y=Gu\pm Gv\). 
Wird die Summierung \emph{hinter} \(G\) platziert, muss der zweite Zweig ebenfalls mit \(G\) gewichtet werden.
\begin{center}
\begin{tikzpicture}[node distance=12mm]
  \node (u) {$u$};
  \node[sum, right=of u] (s) {};
  \node[block, right=of s] (g) {$G$};
  \node[right=of g] (y) {$y$};
  \node (v) [below=10mm of s] {$v$};

  \draw[line] (u) -- (s) -- (g) -- (y);
  \draw[line] (v) -- (s);

  \node at ($(s)+(-3.5mm,2.2mm)$) {$+$};
  \node at ($(s)+(-3.5mm,-2.2mm)$) {$\pm$};
\end{tikzpicture}
\qquad$\Longleftrightarrow$\qquad
\begin{tikzpicture}[node distance=12mm]
  \node (u) {$u$};
  \node[block, right=of u] (g1) {$G$};
  \node[sum, right=of g1] (s) {};
  \node[right=of s] (y) {$y$};
  \node (v) [below=10mm of s] {$v$};
  \node[block, left=of v] (g2) {$G$};

  \draw[line] (u) -- (g1) -- (s) -- (y);
  \draw[line] (v) -- (g2) -- (s);

  \node at ($(s)+(-3.5mm,2.2mm)$) {$+$};
  \node at ($(s)+(-3.5mm,-2.2mm)$) {$\pm$};
\end{tikzpicture}
\end{center}

\paragraph{(b) Summierstelle \(\leftarrow\) vor \(\boldsymbol{G}\) verschieben:}
Umgekehrt gilt: Aus \(y=Gu\pm v\) wird \(y=G\bigl(u\pm v/G\bigr)\). 
Beim Verschieben \emph{vor} \(G\) muss der Zweig, der \(G\) zuvor \emph{nicht} durchlief, mit \(1/G\) skaliert werden.

\subsubsection{Regel 5: Abzweigstelle (Messabgriff) über einen Block verschieben}
Auch hier gilt: das abgezweigte Signal muss gleich bleiben.

\paragraph{(a) Abzweig \(\rightarrow\) nach \(\boldsymbol{G}\) verschieben:}
Wird hinter \(G\) abgegriffen, ist das Signal um \(G\) größer; daher muss im Abzweig ein Faktor \(1/G\) eingefügt werden, um das ursprüngliche Signal beizubehalten.
\begin{center}
\begin{tikzpicture}[node distance=12mm]
  \node (u) {$u$};
  \node[branch, right=of u] (b) {};
  \node[block, right=of b] (g) {$G$};
  \node[right=of g] (y) {$y$};
  \node[below=10mm of b] (tap) {$u$};

  \draw[line] (u) -- (b) -- (g) -- (y);
  \draw[line] (b) -- (tap);
\end{tikzpicture}
\qquad$\Longleftrightarrow$\qquad
\begin{tikzpicture}[node distance=12mm]
  \node (u) {$u$};
  \node[block, right=of u] (g) {$G$};
  \node[branch, right=of g] (b) {};
  \node[right=of b] (y) {$y$};
  \node[below=10mm of b] (tap) {$u$};
  \node[block, left=of tap] (inv) {$1/G$};

  \draw[line] (u) -- (g) -- (b) -- (y);
  \draw[line] (b) -- (inv) -- (tap);
\end{tikzpicture}
\end{center}

\paragraph{(b) Abzweig \(\leftarrow\) vor \(\boldsymbol{G}\) verschieben:}
Analog: Beim Verschieben eines Abgriffs \emph{vor} \(G\) wird im Abzweig ein Faktor \(G\) benötigt.

\subsubsection{Regel 6: Innere Schleifen zuerst reduzieren}
Bei verschachtelten Rückführungen ist es oft am einfachsten,
\begin{itemize}
  \item zunächst \emph{innere} Feedback-Schleifen mit Regel~3 zu einem Ersatzblock zusammenzufassen,
  \item danach Serien- und Parallelblöcke (Regeln~1--2) zu reduzieren,
  \item und zuletzt äußere Schleifen zu schließen.
\end{itemize}

\paragraph{Praxis-Tipp:} Zeichne bei jeder Umformung kurz die \emph{Signalgleichungen} an den Knoten (z.\,B.\ \(y=Gu\), \(e=w-x\)). 
Wenn die Gleichungen identisch bleiben, ist die Umformung korrekt.


\end{document}
