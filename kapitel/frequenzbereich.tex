% kapitel/frequenzbereich.tex
\section{Beschreibung im Frequenzbereich}
\subsection{Frequenzgang}
F\"ur sinusf\"ormige Signale setzt man $p=\jj\omega$ und erh\"alt den Frequenzgang:
\[
  G(\jj\omega)=G(p)\big|_{p=\jj\omega}.
\]
Schreibweisen:
\[
  G(\jj\omega)=\Ree(\omega)+\jj\,\Imm(\omega)
  \quad\text{oder}\quad
  G(\jj\omega)=\lvert G(\jj\omega)\rvert\,\mathrm{e}^{\jj\varphi(\omega)}.
\]
Betrag und Phase:
\[
  \lvert G(\jj\omega)\rvert=\sqrt{\Ree(\omega)^2+\Imm(\omega)^2},
  \qquad
  \varphi(\omega)=\argg\!\bigl(G(\jj\omega)\bigr).
\]
Bei der Winkelbestimmung ist auf den Quadranten zu achten (praktisch: \texttt{atan2}).

\paragraph{Rechenregel (komplexe Br\"uche).}
F\"ur $G(\jj\omega)=\frac{Z(\omega)}{N(\omega)}$ mit $N(\omega)\neq 0$:
\[
  \frac{Z}{N}=\frac{Z\,\overline{N}}{N\,\overline{N}},
\]
woraus sich $\Ree,\Imm$ bequem bestimmen lassen.

\paragraph{Beispiel (PT1, $K=1$).}
Mit $G(\jj\omega)=\frac{1}{1+\jj\omega T}$ folgt
\[
  G(\jj\omega)=\frac{1-\jj\omega T}{1+(\omega T)^2},
  \quad
  \Ree(\omega)=\frac{1}{1+(\omega T)^2},\ 
  \Imm(\omega)=\frac{-\omega T}{1+(\omega T)^2}.
\]

\subsection{Ortskurve}
Die Ortskurve ist der geometrische Ort aller Werte von $G(\jj\omega)$ in der komplexen Ebene,
wenn $\omega$ von $0$ bis $\infty$ l\"auft.
