% kapitel/frequenzbereich.tex
\section{Beschreibung im Frequenzbereich}
\subsection{Frequenzgang}
F\"ur sinusf\"ormige Signale setzt man $p=\jj\omega$ und erh\"alt den Frequenzgang:
\[
  G(\jj\omega)=G(p)\big|_{p=\jj\omega}.
\]
Hinweis: Die R\"uckrichtung gilt nicht ohne Weiteres; aus $G(\jj\omega)$ folgt kein
eindeutiges $G(p)$ im Zeitbereich/Laplace-Bereich ohne zus\"atzliche Annahmen.
Schreibweisen:
\[
  G(\jj\omega)=\Ree(\omega)+\jj\,\Imm(\omega)
  \quad\text{oder}\quad
  G(\jj\omega)=\lvert G(\jj\omega)\rvert\,\mathrm{e}^{\jj\varphi(\omega)}.
\]
Betrag und Phase:
\[
  \lvert G(\jj\omega)\rvert=\sqrt{\Ree(\omega)^2+\Imm(\omega)^2},
  \qquad
  \varphi(\omega)=\argg\!\bigl(G(\jj\omega)\bigr).
\]
Bei der Winkelbestimmung ist auf den Quadranten zu achten (praktisch: \texttt{atan2}).

\paragraph{Rechenregel (komplexe Br\"uche).}
F\"ur $G(\jj\omega)=\frac{Z(\omega)}{N(\omega)}$ mit $N(\omega)\neq 0$:
\[
  \frac{Z}{N}=\frac{Z\,\overline{N}}{N\,\overline{N}},
\]
woraus sich $\Ree,\Imm$ bequem bestimmen lassen.

\paragraph{Beispiel (PT1, $K=1$).}
Mit $G(\jj\omega)=\frac{1}{1+\jj\omega T}$ folgt
\[
  G(\jj\omega)=\frac{1-\jj\omega T}{1+(\omega T)^2},
  \quad
  \Ree(\omega)=\frac{1}{1+(\omega T)^2},\ 
  \Imm(\omega)=\frac{-\omega T}{1+(\omega T)^2}.
\]

\subsection{Ortskurve}
Die Ortskurve ist der geometrische Ort aller Werte von $G(\jj\omega)$ in der komplexen Ebene,
wenn $\omega$ von $0$ bis $\infty$ l\"auft.
Man konstruiert sie, indem $\omega$ als Parameter durchlaufen wird und die Punkte
$(\Ree(\omega),\Imm(\omega))$ eingezeichnet werden; f\"ur Systeme mit reellen Koeffizienten
ergibt sich die vollst\"andige Kurve durch Spiegelung an der reellen Achse.
Kernaussage (Pr\"ufungsstoff): Der Verlauf macht Betrag und Phase im Frequenzbereich sichtbar
und bildet die Grundlage f\"ur Nyquist-Betrachtungen (Stabilit\"at, Reserven).
Die zugeh\"origen Nyquist-Ortskurven der gezeichneten Standardglieder (Normierung $K=1$, $T=1$)
sind in Tabelle~\ref{tab:nyquist_standardglieder_formeln} zusammengefasst; daneben steht die
allgemeine Form. Die gezeichneten PT$n$ entsprechen kaskadierten PT1 mit gleichen Zeitkonstanten.
\begin{table}[H]
  \centering
  \begin{tabular}{@{}lll@{}}
    \toprule
    Glied & Gezeichnet ($K=1$, $T=1$) & Allgemein \\
    \midrule
    P   & $G(\jj\omega)=1$ & $G(\jj\omega)=K$ \\
    I   & $G(\jj\omega)=\dfrac{1}{\jj\omega}$ & $G(\jj\omega)=\dfrac{K_I}{\jj\omega}$ \\
    D   & $G(\jj\omega)=\jj\omega$ & $G(\jj\omega)=K_D\,\jj\omega$ \\
    PI  & $G(\jj\omega)=1+\dfrac{1}{\jj\omega}$ & $G(\jj\omega)=K_P\!\left(1+\dfrac{1}{T_I\,\jj\omega}\right)$ \\
    PD  & $G(\jj\omega)=1+\jj\omega$ & $G(\jj\omega)=K_P\!\left(1+T_D\,\jj\omega\right)$ \\
    PT1 & $G(\jj\omega)=\dfrac{1}{1+\jj\omega}$ & $G(\jj\omega)=\dfrac{K}{1+T\,\jj\omega}$ \\
    PT2 & $G(\jj\omega)=\dfrac{1}{(1+\jj\omega)^2}$ & $G(\jj\omega)=\dfrac{K}{1+2D\,T\,\jj\omega+(T\,\jj\omega)^2}$ \\
    PT3 & $G(\jj\omega)=\dfrac{1}{(1+\jj\omega)^3}$ & -- \\
    PT4 & $G(\jj\omega)=\dfrac{1}{(1+\jj\omega)^4}$ & -- \\
    PT7 & $G(\jj\omega)=\dfrac{1}{(1+\jj\omega)^7}$ & -- \\
    \bottomrule
  \end{tabular}
  \caption{Nyquist-Ortskurven der gezeichneten Standardglieder und allgemeine Formen.}
  \label{tab:nyquist_standardglieder_formeln}
\end{table}
F\"ur kaskadierte PT$N$-Glieder mit unterschiedlichen Zeitkonstanten gilt allgemein
\[
  G(\jj\omega)=\frac{K}{\prod_{i=1}^n \bigl(1+T_i\,\jj\omega\bigr)}.
\]
Die gezeichnete PT2-Kurve entspricht dem Sonderfall $D=1$ (kritische D\"ampfung) bei $T=1$.

% Nyquist curves of standard elements (K=1, T=1)
\begin{figure}[H]
  \centering
  \begin{tikzpicture}
    \begin{axis}[
      width=0.9\linewidth,
      height=0.6\linewidth,
      axis lines=middle,
      axis equal image,
      xmin=-0.6, xmax=1.1,
      ymin=-1.2, ymax=1.2,
      grid=both,
      grid style={line width=.1pt, draw=gray!20},
      major grid style={line width=.2pt, draw=gray!35},
      tick label style={font=\small},
      label style={font=\small},
      xlabel={$\Ree$},
      ylabel={$\Imm$},
      legend style={font=\small, at={(0.0,1.02)}, anchor=south west, legend columns=3},
    ]
      \addplot[thick, blue] table[x=re,y=im,col sep=comma] {data/nyquist_PT1_K1_T1.csv};
      \addlegendentry{PT1}
      \addplot[thick, red] table[x=re,y=im,col sep=comma] {data/nyquist_PT2_K1_T1.csv};
      \addlegendentry{PT2}
      \addplot[thick, green!60!black] table[x=re,y=im,col sep=comma] {data/nyquist_PT3_K1_T1.csv};
      \addlegendentry{PT3}
      \addplot[thick, orange] table[x=re,y=im,col sep=comma] {data/nyquist_PT4_K1_T1.csv};
      \addlegendentry{PT4}
      \addplot[thick, teal!70!black] table[x=re,y=im,col sep=comma] {data/nyquist_PT7_K1_T1.csv};
      \addlegendentry{PT7}
      \addplot[thick, gray, dashed, restrict y to domain=-1.2:1.2] table[x=re,y=im,col sep=comma] {data/nyquist_I_K1.csv};
      \addlegendentry{I}
      \addplot[thick, black, dotted, restrict y to domain=-1.2:1.2] table[x=re,y=im,col sep=comma] {data/nyquist_D_K1.csv};
      \addlegendentry{D}
      \addplot[thick, cyan!60!black, dashdotdotted, restrict y to domain=-1.2:1.2] table[x=re,y=im,col sep=comma] {data/nyquist_PI_K1_T1.csv};
      \addlegendentry{PI}
      \addplot[thick, brown, dashed, restrict y to domain=-1.2:1.2] table[x=re,y=im,col sep=comma] {data/nyquist_PD_K1_T1.csv};
      \addlegendentry{PD}
      \addplot[only marks, mark=*, mark size=1.6pt] coordinates {(1,0)};
      \addlegendentry{P}
    \end{axis}
  \end{tikzpicture}
  \caption{Nyquist-Ortskurven der Standardglieder (Ausschnitt, $K=1$, $T=1$; PT1--PT4, PT7,
  PI und PD). Gezeigt ist
  $\omega>0$; die vollst\"andige Kurve ergibt sich durch Spiegelung an der reellen Achse.
  Das I- und D-Glied verlaufen entlang der imagin\"aren Achse.}
  \label{fig:nyquist_standardglieder}
\end{figure}

% Nyquist curves of PT elements (zoom around origin)
\begin{figure}[H]
  \centering
  \begin{tikzpicture}
    \begin{axis}[
      width=0.82\linewidth,
      height=0.55\linewidth,
      axis lines=middle,
      axis equal image,
      xmin=-0.3, xmax=0.3,
      ymin=-0.3, ymax=0.3,
      grid=both,
      grid style={line width=.1pt, draw=gray!20},
      major grid style={line width=.2pt, draw=gray!35},
      tick label style={font=\small},
      label style={font=\small},
      xlabel={$\Ree$},
      ylabel={$\Imm$},
      legend style={font=\small, at={(0.02,0.98)}, anchor=north west},
    ]
      \addplot[thick, blue] table[x=re,y=im,col sep=comma] {data/nyquist_PT1_K1_T1.csv};
      \addlegendentry{PT1}
      \addplot[thick, red] table[x=re,y=im,col sep=comma] {data/nyquist_PT2_K1_T1.csv};
      \addlegendentry{PT2}
      \addplot[thick, green!60!black] table[x=re,y=im,col sep=comma] {data/nyquist_PT3_K1_T1.csv};
      \addlegendentry{PT3}
      \addplot[thick, orange] table[x=re,y=im,col sep=comma] {data/nyquist_PT4_K1_T1.csv};
      \addlegendentry{PT4}
      \addplot[thick, teal!70!black] table[x=re,y=im,col sep=comma] {data/nyquist_PT7_K1_T1.csv};
      \addlegendentry{PT7}
      \filldraw (0,0) circle (1.2pt);
    \end{axis}
  \end{tikzpicture}
  \caption{Zoom um den Ursprung: Nyquist-Ortskurven der PT-Glieder ($K=1$, $T=1$; PT1--PT4 und PT7).
  Mit wachsendem $n$ verl\"auft die Kurve n\"aher um $(0,0)$ und durchl\"auft dabei mehrere
  Quadranten (nur $\omega>0$, Spiegelung an der reellen Achse ergibt die vollst\"andige Kurve).}
  \label{fig:nyquist_standardglieder_zoom0}
\end{figure}

