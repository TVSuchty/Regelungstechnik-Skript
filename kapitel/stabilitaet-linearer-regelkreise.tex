% kapitel/stabilitaet-linearer-regelkreise.tex
\section{Stabilit\"at linearer Regelkreise}
\subsection{Charakteristische Gleichung}
Ein LZI-System ist (asymptotisch) stabil, wenn alle Pole strikt in der linken Halbebene liegen
($\Ree(p_i)<0$). F\"ur den einschleifigen Regelkreis folgt die charakteristische Gleichung aus dem Nenner:
\[
  1+G_0(p)=0.
\]

\subsection{Nyquist-Idee (qualitativ)}
F\"ur sinusf\"ormige Anregung sind im eingeschwungenen Zustand alle Signale sinusf\"ormig
(mit anderer Amplitude und Phasenlage). Mit $\omega_k$ bezeichnen wir eine beliebige abgetastete
Frequenz. Wird bei einer Frequenz $\omega_k$ eine Phasendrehung
von $\varphi(\omega_k)=-180^\circ$ erreicht, wirkt die R\"uckkopplung wie positive R\"uckkopplung.
Dann entscheidet der Betrag:
\[
  \lvert G_0(\jj\omega_k)\rvert
  \begin{cases}
    <1 & \Rightarrow \text{Schwingungen klingen ab (stabil)},\\
    =1 & \Rightarrow \text{Grenzstabilit\"at (Dauerschwingung)},\\
    >1 & \Rightarrow \text{Schwingungen wachsen an (instabil)}.
  \end{cases}
\]
F\"ur Systeme ohne Pole in der rechten Halbebene kann man in vielen Anwendungen die vereinfachte
Schnittbedingung nutzen: An Frequenzen mit $\Imm(G_0(\jj\omega))=0$ muss
$\Ree(G_0(\jj\omega))>-1$ gelten.

\paragraph{Warum die Bedingung $\Ree(G_0(\jj\omega))>-1$?}
Auf der reellen Achse ($\Imm=0$) hat $G_0(\jj\omega)$ die Phasenlage $\varphi=0^\circ$ oder
$-180^\circ$. Erreicht die Kurve dort $\Ree\le -1$, dann liegt der kritische Punkt $-1+\jj0$ auf
oder links des aktuellen Kurvenpunktes. Das entspricht Betrag $\lvert G_0\rvert\ge 1$ bei
$\varphi=-180^\circ$ und damit genau der Grenze (bzw.\ dem \"Ubertritt) von stabil zu instabil:
$1+G_0(\jj\omega)=0$ ist erf\"ullt (der Punkt $-1+\jj0$ wird getroffen), oder $-1$ wird umschlungen.
Liegt der Realteil dagegen gr\"o\ss er als $-1$, befindet sich die Ortskurve auf der ``stabilen''
Seite des kritischen Punkts, d.\,h.\ bei $-180^\circ$ ist der Betrag kleiner als 1 und der
Nyquist-Test bleibt erf\"ullt.

\begin{center}
  \begin{tikzpicture}[scale=1.0]
    % axes
    \draw[->,thin] (-2.5,0) -- (1.5,0) node[right] {$\Ree$};
    \draw[->,thin] (0,-1.4) -- (0,1.4) node[above] {$\Imm$};
    % critical point -1
    \filldraw (-1,0) circle (1.2pt);
    \node[below left] at (-1,0) {$-1$};
    % stable crossing (Re>-1)
    \draw[line width=0.9pt,blue] plot[smooth] coordinates {(-0.6,-1.0) (-0.7,-0.4) (-0.8,0) (-0.7,0.5) (-0.4,1.0)};
    \node[blue,align=left] at (-0.15,1.05) {\footnotesize stabiler Schnitt:\\[-1pt] $|G_0|<1$ bei $-180^\circ$};
    % unstable crossing (Re<=-1)
    \draw[line width=0.9pt,red,dashed] plot[smooth] coordinates {(-1.4,-1.0) (-1.2,-0.4) (-1.1,0) (-1.2,0.5) (-1.5,1.0)};
    \node[red,align=left] at (-1.95,1.05) {\footnotesize instabiler Schnitt:\\[-1pt] $|G_0|\ge 1$ bei $-180^\circ$};
    % arrows showing direction (increasing omega)
    \draw[blue,-Latex] (-0.74,-0.1) -- (-0.78,0.15);
    \draw[red,-Latex] (-1.16,-0.1) -- (-1.20,0.15);
  \end{tikzpicture}
\end{center}
Die blaue Kurve schneidet die reelle Achse rechts von $-1$ ($|G_0|<1$ bei $-180^\circ$) und umschlingt
$-1$ nicht: der geschlossene Kreis bleibt stabil. Die rote, gestrichelte Kurve kreuzt links von $-1$
($|G_0|\ge 1$ bei $-180^\circ$); damit wird $-1$ getroffen oder umschlungen und mindestens ein
geschlossener Pol wandert in die rechte Halbebene (Instabilit\"at).

\subsection{Nyquist-Diagramm und Nyquist-Kriterium}
\paragraph{Nyquist-Diagramm (Nyquist-Ortskurve).}
F\"ur die Kreisverst\"arkung (aufgeschnittener Regelkreis)
\[
  G_0(p)=G_R(p)\,G_S(p)
\]
hei\ss t \emph{Nyquist-Diagramm} die Ortskurve der komplexen Werte $G_0(\jj\omega)$
in der $\Ree$--$\Imm$-Ebene, wenn $\omega$ von $0$ bis $\infty$ l\"auft.
F\"ur Systeme mit reellen Koeffizienten ergibt sich die vollst\"andige Nyquist-Kurve
durch Spiegelung an der reellen Achse.

\paragraph{Stabilit\"atsfrage \"uber $1+G_0(p)=0$.}
F\"ur den einschleifigen Regelkreis mit Einheitsr\"uckf\"uhrung ist die charakteristische Gleichung
\[
  1+G_0(p)=0.
\]
Die Nullstellen von $1+G_0(p)$ sind die Pole des geschlossenen Kreises
(Stabilit\"at $\Leftrightarrow$ alle Pole in der linken Halbebene).

\paragraph{Nyquist-Kriterium (Vorzeichenkonvention).}
Sei $P$ die Anzahl der Pole von $G_0(p)$ in der rechten Halbebene und $Z$ die Anzahl der
Nullstellen von $1+G_0(p)$ in der rechten Halbebene (instabile geschlossene Pole).
Weiter sei $N$ die Anzahl der \emph{Uhrzeigersinn}-Umschlingungen des Punktes $-1+\jj0$
durch die vollst\"andige Nyquist-Kurve von $G_0(\jj\omega)$. Dann gilt
\[
  N=Z-P
  \qquad\Leftrightarrow\qquad
  Z=N+P.
\]
Damit folgt unmittelbar:
\[
  \text{geschlossener Kreis stabil} \;\;\Leftrightarrow\;\; Z=0 \;\;\Leftrightarrow\;\; N=-P.
\]
Insbesondere: Ist der aufgeschnittene Kreis bereits stabil ($P=0$), dann ist der
geschlossene Kreis genau dann stabil, wenn die Nyquist-Kurve $-1$ \emph{nicht}
im Uhrzeigersinn umschlingt ($N=0$).
Ein formaler Beweis des Nyquist-Kriteriums (über die Argument Principle) wird hier nicht geführt;
wir nutzen das Ergebnis als praktisches Stabilitätswerkzeug.

\paragraph{Beispiel.}
Abbildung~\ref{fig:nyquist_example} zeigt
$G_0(p)=\dfrac{K}{(1+p)(1+0{,}2p)(1+0{,}05p)}$.
Bei $K\approx31{,}5$ verl\"auft die Ortskurve durch $-1$ (Grenzstabilit\"at);
f\"ur gr\"o\ss eres $K$ entstehen instabile Pole im geschlossenen Kreis.
\begin{itemize}[itemsep=2pt]
  \item $K=20$ (durchgezogene Linie): Ortskurve bleibt rechts von $-1$ und umschlingt $-1$ nicht
    ($N=0$, $P=0 \Rightarrow Z=0$) \(\Rightarrow\) stabil.
  \item $K\approx31{,}5$ (gestrichelt): Ortskurve l\"auft durch $-1$ \(\Rightarrow\) Grenzfall, der
    geschlossene Kreis besitzt Pole auf der imagin\"aren Achse (Dauerschwingung).
  \item $K=40$ (punktiert): Ortskurve umschlingt $-1$ im Uhrzeigersinn einmal
    ($N=1$, $P=0 \Rightarrow Z=1$) \(\Rightarrow\) ein instabiler Pol im geschlossenen Kreis.
\end{itemize}

% Neue Abbildung 7 (Zoom) aus Abbildung7_neu_Paket_ZOOM
\begin{figure}[tb]
  \centering
  \includegraphics[width=0.8\linewidth]{Abbildung7_neu_Paket_ZOOM/Abbildung7_neu_latex.pdf}
  \caption{Nyquist-Ortskurven (Zoom um $(-1,0)$) der Kreisverstärkung
  $G_0(p)=\dfrac{K}{(1+p)(1+0{,}2p)(1+0{,}05p)}$ (nur $\omega>0$; vollständige Kurve durch Spiegelung).
  Stabil: $K=20$ bleibt rechts und berührt $(-1,0)$ nicht.
  Grenzstabil: $K\approx 31{,}50$ trifft $(-1,0)$.
  Instabil: $K=40$ umschlingt $(-1,0)$ im Uhrzeigersinn.}
  \label{fig:nyquist_example}
\end{figure}


\paragraph{Praktische Gr\"o\ss en: Verst\"arkungs- und Phasenreserve.}
Aus dem Nyquist- bzw.\ Bode-Diagramm lassen sich Stabilit\"atsreserven ablesen:
\begin{itemize}[itemsep=2pt]
  \item \textbf{Durchtrittsfrequenz} $\omega_c$: $|G_0(\jj\omega_c)|=1$.
    \textbf{Phasenreserve} $\varphi_m = 180^\circ + \arg(G_0(\jj\omega_c))$.
  \item \textbf{Phasendurchtritt} $\omega_\pi$: $\arg(G_0(\jj\omega_\pi))=-180^\circ$ (falls vorhanden).
    \textbf{Verst\"arkungsreserve} $G_m = 1/|G_0(\jj\omega_\pi)|$
    (in dB: $20\log_{10}G_m$).
\end{itemize}
Gro\ss e Reserven bedeuten typischerweise: Die Nyquist-Kurve bleibt deutlich vom kritischen
Punkt $-1$ entfernt.

\subsection{Hurwitz-Kriterium (Polynomkriterium)}
Bringt man die charakteristische Gleichung in die Polynomform
\[
  a_0+a_1p+a_2p^2+\cdots+a_np^n=0,
\]
dann ist eine notwendige Bedingung f\"ur Stabilit\"at $a_i>0$ f\"ur alle $i$.
Das Hurwitz-Kriterium liefert eine notwendige und hinreichende Bedingung:
Alle Hurwitz-Determinanten m\"ussen positiv sein.

\paragraph{Hurwitz-Matrix (Schema).}
\[
  H=
  \begin{pmatrix}
    a_1 & a_3 & a_5 & a_7 & \cdots\\
    a_0 & a_2 & a_4 & a_6 & \cdots\\
    0   & a_1 & a_3 & a_5 & \cdots\\
    0   & a_0 & a_2 & a_4 & \cdots\\
    \vdots & \vdots & \vdots & \vdots & \ddots
  \end{pmatrix}
\]
Die Hurwitz-Determinanten $H_k$ sind die Determinanten der linken oberen $k\times k$-Untermatrix, z.\,B.:
\[
  H_1 = \begin{vmatrix} a_1 \end{vmatrix} = a_1,\qquad
  H_2 = \begin{vmatrix} a_1 & a_3 \\[4pt] a_0 & a_2 \end{vmatrix} = a_1 a_2 - a_0 a_3,
  \qquad
  H_3 = \begin{vmatrix}
    a_1 & a_3 & a_5 \\
    a_0 & a_2 & a_4 \\
    0   & a_1 & a_3
  \end{vmatrix}.
\]
Im Allgemeinen gilt: $H_k$ ist die Determinante der $k\times k$-Untermatrix in der linken oberen Ecke von $H$.
Stabilit\"at $\Leftrightarrow H_k>0$ f\"ur alle $k=1,\ldots,n$.

\paragraph{Praktische Kurzform ($n=3$).}
\[
  a_0>0,\ a_1>0,\ a_2>0,\ a_3>0
  \quad\text{und}\quad
  a_1a_2>a_0a_3.
\]
Diese Kurzform ist genau die Auswertung von $H_1>0$, $H_2>0$ und $H_3>0$ f\"ur ein Polynom 3.\ Ordnung:
Positive Koeffizienten stellen $H_1$ sicher, $a_1a_2>a_0a_3$ entspricht $H_2>0$, und bei drei Polynomen
gen\"ugen damit alle Bedingungen des Hurwitz-Kriteriums.

\paragraph{Beispiel ($n=2$).}
F\"ur $p^2+2p+5=0$ gilt $a_2=1$, $a_1=2$, $a_0=5$.
Alle Koeffizienten sind positiv, daher ist das System stabil.

\subsection{Beispiel: PT2-Strecke mit PI-Regler (Stabilit\"atsgebiet)}
Sei
\[
  G_S(p)=\frac{K_S}{1+2DTp+T^2p^2},
  \qquad
  G_R(p)=K_{PR}+\frac{K_{IR}}{p}.
\]
Dann ist
\[
  G_0(p)=G_R(p)G_S(p)=\left(K_{PR}+\frac{K_{IR}}{p}\right)\frac{K_S}{1+2DTp+T^2p^2}.
\]
Charakteristische Gleichung:
\[
  1+G_0(p)=0
  \ \Leftrightarrow\
  p(1+2DTp+T^2p^2)+K_S K_{PR}\,p + K_S K_{IR}=0.
\]
Damit ergibt sich das Stabilit\"atspolynom 3.\ Ordnung:
\[
  a_3p^3+a_2p^2+a_1p+a_0=0
\]
mit
\[
  a_3=T^2,\quad a_2=2DT,\quad a_1=1+K_S K_{PR},\quad a_0=K_S K_{IR}.
\]
Hurwitz ($n=3$) liefert:
\[
  K_{IR}>0,\qquad
  1+K_S K_{PR}>0\ \Leftrightarrow\ K_{PR}>-\frac{1}{K_S},
\]
\[
  (1+K_S K_{PR})\,2DT > (K_S K_{IR})\,T^2
  \ \Leftrightarrow\
  K_{IR}<\frac{2D}{K_S T}\,(1+K_S K_{PR}).
\]
Diese Ungleichungen sind exakt die Bedingungen $H_1>0$, $H_2>0$, $H_3>0$ des Hurwitz-Kriteriums
f\"ur das Polynom 3.\ Ordnung: $H_1>0$ fordert $a_1>0$ ($\Rightarrow K_{PR}>-1/K_S$), $a_0>0$ liefert
$K_{IR}>0$, und $H_2>0$ ergibt die Schranke $K_{IR}<\tfrac{2D}{K_S T}(1+K_S K_{PR})$.
Damit l\"asst sich das Stabilit\"atsgebiet im $(K_{PR},K_{IR})$-Diagramm grafisch darstellen.

\paragraph{Beispielhafte Parameterwahl.}
F\"ur $K_S=1$, $D=0{,}5$, $T=1$ wird daraus:
\[
  K_{IR}>0,\qquad K_{PR}>-1,\qquad K_{IR}<1+K_{PR}.
\]
Damit ist das Stabilit\"atsgebiet ein einfaches Dreieck (Abbildung~\ref{fig:stability_region_pi_pt2}).

% Stability region for PT2 plant with PI controller (example parameters)
\begin{figure}[ht]
  \centering
  \begin{tikzpicture}
    \begin{axis}[
      width=0.70\textwidth, height=0.45\textwidth,
      grid=both, grid style={line width=.1pt, draw=gray!20},
      major grid style={line width=.2pt, draw=gray!35},
      xlabel={$K_{PR}$}, ylabel={$K_{IR}$},
      xmin=-1.2, xmax=5.2,
      ymin=-0.2, ymax=6.2,
      axis lines=left,
      tick label style={font=\small},
      label style={font=\small},
    ]
      % Filled stability region polygon
      \addplot[fill=gray!15, draw=none] table[x=KPR,y=KIR,col sep=comma] {data/stab_region_polygon_PI_PT2_KS1_D0p5_T1.csv} \closedcycle;
      % Boundary lines
      \addplot[thick] table[x=KPR,y=KIR_max,col sep=comma] {data/stab_region_PI_PT2_KS1_D0p5_T1.csv};
      \addplot[thick] coordinates {(-1,0) (5,0)};
      \addplot[dashed] coordinates {(-1,0) (-1,6)};
      \node[font=\small, anchor=west] at (axis cs:-0.95,5.55) {$K_S=1,\;D=0{,}5,\;T=1$};
      \node[font=\small, anchor=west] at (axis cs:-0.95,4.95) {$0<K_{IR}<1+K_{PR},\;\;K_{PR}>-1$};
    \end{axis}
  \end{tikzpicture}
  \caption{Beispielhaftes Stabilit\"atsgebiet f\"ur eine PT2-Strecke mit PI-Regler (Parameterwahl $K_S=1$, $D=0{,}5$, $T=1$). Das schattierte Gebiet erf\"ullt die Hurwitz-Bedingungen $H_1,H_2,H_3$ aus Abschnitt~11.4.}
  \label{fig:stability_region_pi_pt2}
\end{figure}

