% kapitel/typische-signale.tex
\section{Typische Signale}
\subsection{Aperiodische Eingangssignale}
\paragraph{Sprungfunktion (Einheitssprung).}
\[
  \sigma(t)=
  \begin{cases}
    0, & t<0,\\
    1, & t\ge 0.
  \end{cases}
\]
Ein allgemeiner Sprung der H\"ohe $x_{e0}$ lautet $x_e(t)=x_{e0}\,\sigma(t)$.

\paragraph{Rechteckimpuls.}
Ein Rechteckimpuls (H\"ohe $x_{e0}$, Dauer $t_0$) l\"asst sich als Differenz zweier Spr\"unge schreiben:
\[
  x_e(t)=x_{e0}\bigl(\sigma(t)-\sigma(t-t_0)\bigr).
\]

\paragraph{Beispiel (Rechteckimpuls).}
F\"ur $x_{e0}=2$ und $t_0=0{,}5$ gilt
$x_e(t)=2\bigl(\sigma(t)-\sigma(t-0{,}5)\bigr)$.

\paragraph{Rampe.}
Die (Einheits-)Rampe ist das Integral der Sprungfunktion:
\[
  r(t)=\int_{0}^{t}\sigma(\tau)\,\dd\tau = t\,\sigma(t).
\]


% Typical input signals (step, ramp, rectangular pulse)
\begin{figure}[tb]
  \centering
  \begin{tikzpicture}
    \begin{groupplot}[
      group style={group size=2 by 2, horizontal sep=1.4cm, vertical sep=1.0cm},
      width=0.46\textwidth, height=0.28\textwidth,
      grid=both, grid style={line width=.1pt, draw=gray!20},
      major grid style={line width=.2pt, draw=gray!35},
      xlabel={$t$}, xmin=-0.5, xmax=2,
      ymin=-0.1, ymax=2.1,
      axis lines=left,
      tick label style={font=\small},
      label style={font=\small},
      title style={font=\small},
    ]
      \nextgroupplot[title={Sprung $\sigma(t)$}, ylabel={$x_e(t)$}]
        \addplot[thick] table[x=t,y=step,col sep=comma] {data/signals_step_ramp_pulse.csv};
        \addplot[only marks, mark=*, mark size=1.2pt] coordinates {(0,0)}; % discontinuity marker
      \nextgroupplot[title={Rampe $r(t)=t\,\sigma(t)$}]
        \addplot[thick] table[x=t,y=ramp,col sep=comma] {data/signals_step_ramp_pulse.csv};
      \nextgroupplot[title={Rechteckimpuls $\sigma(t)-\sigma(t-t_0)$}]
        \addplot[thick] table[x=t,y=pulse,col sep=comma] {data/signals_step_ramp_pulse.csv};
        \addplot[dashed] coordinates {(1,0) (1,1)};
      \nextgroupplot[title={Diracimpuls $\delta(t)$}]
        \addplot[thick, ->] coordinates {(0,0) (0,1.6)};
        \addplot[only marks, mark=*, mark size=1.2pt] coordinates {(0,0)};
        \node[anchor=west, font=\small] at (axis cs:0.05,1.55) {$\delta(t)$};
    \end{groupplot}
  \end{tikzpicture}
  \caption{Typische Eingangssignale im Zeitbereich (normiert, $t_0=1$) inklusive Diracimpuls.}
  \label{fig:signals_typical}
\end{figure}








\subsection{Impulsfunktion (Dirac-Sto\ss)}
Die Dirac-Impulsfunktion $\delta(t)$ ist eine idealisierte Grenzfunktion; im strengen mathematischen Sinn ist $\delta(t)$ keine (klassische) Funktion, sondern eine Distribution (verallgemeinerte Funktion).
Sie wird als Idealisierung eines sehr kurzen Impulses mit endlichem Fl\"acheninhalt verstanden und besitzt die Eigenschaften
\[
  \int_{-\infty}^{\infty}\delta(t)\,\dd t = 1,
  \qquad
  \delta(t)=0\ \text{f\"ur}\ t\neq 0.
\]
Wichtige Beziehung zur Sprungfunktion:
\[
  \frac{\dd}{\dd t}\sigma(t)=\delta(t).
\]

\subsection{Antwortfunktionen eines LZI-\"UTG}
\begin{itemize}[itemsep=2pt]
  \item \textbf{\"Ubergangsfunktion (Sprungantwort)} $h(t)$: Antwort auf den Einheitssprung
  $x_e(t)=\sigma(t)$, also $x_a(t)=h(t)$.
  \item \textbf{Gewichtsfunktion (Impulsantwort)} $g(t)$: Antwort auf den Dirac-Impuls
  $x_e(t)=\delta(t)$, also $x_a(t)=g(t)$.
\end{itemize}
Wegen $\delta(t)=\frac{\dd}{\dd t}\sigma(t)$ gilt f\"ur LZI-Systeme der zentrale Zusammenhang:
\[
  g(t)=\frac{\dd}{\dd t}h(t)=\dot h(t).
\]
(Umgekehrt: $h(t)=\int_0^t g(\tau)\,\dd\tau$.)


