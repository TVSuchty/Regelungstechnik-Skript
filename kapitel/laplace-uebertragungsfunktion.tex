% kapitel/laplace-uebertragungsfunktion.tex
\section{Laplace-Transformation und \"Ubertragungsfunktion}
\subsection{Exponentialansatz und Laplace-Operator}
Viele f\"ur die Regelungstechnik relevante Signale lassen sich als (komplexe) Exponentialform schreiben:
\[
  x(t)=\hat X\,\mathrm{e}^{p t},
  \qquad
  p=\sigma+\jj\omega.
\]
F\"ur diesen Ansatz gilt:
\[
  \frac{\dd}{\dd t}x(t)=p\,x(t),\qquad
  \frac{\dd^2}{\dd t^2}x(t)=p^2\,x(t),\ \ldots
\]
Die Variable $p$ wirkt im Bildbereich als Differentialoperator.
Mit dem Laplace-Operator ist die Abbildung $\Laplace\{\cdot\}$ gemeint, die ein
Zeitbereichssignal $x(t)$ in seine Laplace-Transformierte $X(p)$ im $p$-Bereich
ueberfuehrt.
F\"ur ein kausales Signal $x(t)$:
\[
  X(p)=\mathcal{L}\{x(t)\}=\int_{0^-}^{\infty} x(t)\,\mathrm{e}^{-pt}\,\dd t.
\]
In der Regelungstechnik wird h\"aufig mit Null-Anfangsbedingungen gearbeitet, so dass Ableitungen
einfach in Multiplikationen \"ubergehen. Das sieht man aus der partiellen Integration:
\[
  \Laplace\{\dot x(t)\}=\int_{0^-}^{\infty}\dot x(t)\,\mathrm{e}^{-pt}\,\dd t
  =\bigl[x(t)\,\mathrm{e}^{-pt}\bigr]_{0^-}^{\infty}
  +p\int_{0^-}^{\infty}x(t)\,\mathrm{e}^{-pt}\,\dd t
  =-x(0^-)+pX(p),
\]
wobei der Randterm $\bigl[x(t)\,\mathrm{e}^{-pt}\bigr]_{0^-}^{\infty}$ der aus der partiellen
Integration stammende Bewertungsanteil an den Integrationsgrenzen ist. Fuer kausale, abklingende
Signale verschwindet der Anteil bei $t\to\infty$, weil $x(t)$ nicht schneller waechst als
$\mathrm{e}^{\sigma t}$ mit $\sigma<\Ree\{p\}$, und es bleibt der Anfangswertanteil $-x(0^-)$.
Analog gilt
\[
  \Laplace\{\ddot x(t)\}=p^2X(p)-p\,x(0^-)-\dot x(0^-).
\]
Bei Null-Anfangsbedingungen folgt damit $\Laplace\{\dot x(t)\}=pX(p)$ und
$\Laplace\{\ddot x(t)\}=p^2X(p)$.

\subsection{Laplace-Transformation des Dirac-Impulses}
Die Dirac-Impulsfunktion wirkt in Integralen als Abtaster (Sieb-Eigenschaft):
\[
  \int_{0^-}^{\infty}\delta(t)\,f(t)\,\dd t = f(0^-).
\]
Damit folgt direkt
\[
  \Laplace\{\delta(t)\}=\int_{0^-}^{\infty}\delta(t)\,\mathrm{e}^{-pt}\,\dd t=1.
\]
Fuer einen verschobenen Impuls gilt entsprechend
\[
  \Laplace\{\delta(t-t_0)\}=\int_{0^-}^{\infty}\delta(t-t_0)\,\mathrm{e}^{-pt}\,\dd t
  =\mathrm{e}^{-p t_0}\quad (t_0>0).
\]

\subsection{\"Ubertragungsfunktion (\"UTF)}
F\"ur ein LZI-\"UTG mit Null-Anfangsbedingungen gilt im Laplace-Bereich:
\[
  G(p)=\frac{X_a(p)}{X_e(p)}.
\]
Dabei sind $X_a(p)=\Laplace\{x_a(t)\}$ und $X_e(p)=\Laplace\{x_e(t)\}$ die Laplace-Transformationen
der Zeitfunktionen.
Die \"Ubertragungsfunktion $G(p)$ beschreibt damit die dynamische Beziehung zwischen Eingang und
Ausgang im $p$-Bereich und ist bei Null-Anfangsbedingungen allein durch das System bestimmt.
Sobald $G(p)$ bekannt ist, folgt f\"ur jede Anregung $X_a(p)=G(p)\,X_e(p)$.

Die zugeh\"orige Differentialgleichung stammt aus dem physikalischen Modell.

\paragraph{Beispiel (PT1, RC-Tiefpass).}
Ein einfacher RC-Tiefpass (Passfilter 1.\ Ordnung) mit Eingangsspannung $U_e(t)$ und
Ausgangsspannung $U_A(t)$ liefert mit dem Knotenstromsatz
\begin{center}
\begin{circuitikz}[european resistors]
  \draw
    (0,0) node[left]{$U_e(t)$}
    to[R, l=$R$] (3,0) node[circ]{}
    to[short] (4.5,0) node[right]{$U_A(t)$}
    (3,0) to[C, l=$C$] (3,-2)
    node[ground]{};
\end{circuitikz}
\end{center}
\[
  i_R=\frac{U_e(t)-U_A(t)}{R},\qquad i_C=C\,\dot U_A(t),\qquad i_R=i_C
\]
und damit
\[
  RC\,\dot U_A(t)+U_A(t)=U_e(t).
\]
Identifiziert man $x_a=U_A$, $x_e=U_e$, $T=RC$ und $K=1$, ergibt sich die PT1-DGL
\[
  T\,\dot x_a + x_a = K\,x_e.
\]
Im Laplace-Bereich folgt
\[
  (Tp+1)X_a(p)=K\,X_e(p)
  \quad\Longrightarrow\quad
  G(p)=\frac{K}{1+Tp}.
\]

\paragraph{Beispiel (Schrittanregung am PT1).}
F\"ur $x_e(t)=\sigma(t)$ gilt $X_e(p)=\frac{1}{p}$ und damit
\[
  X_a(p)=G(p)\,X_e(p)=\frac{K}{p(1+Tp)}.
\]
R\"ucktransformation liefert die Sprungantwort
\[
  x_a(t)=K\left(1-\mathrm{e}^{-t/T}\right)\sigma(t).
\]


% Step response of a PT1 element
\begin{figure}[tb]
  \centering
  \begin{tikzpicture}
    \begin{axis}[
      width=0.62\textwidth, height=0.38\textwidth,
      grid=both, grid style={line width=.1pt, draw=gray!20},
      major grid style={line width=.2pt, draw=gray!35},
      xlabel={$t/T$}, ylabel={$x_a(t)/K$},
      xmin=0, xmax=5,
      ymin=0, ymax=1.05,
      axis lines=left,
      tick label style={font=\small},
      label style={font=\small},
    ]
      \addplot[thick] table[x=t,y=y,col sep=comma] {data/pt1_step_K1_T1.csv};
      \addplot[dashed] coordinates {(0,1) (5,1)};
    \end{axis}
  \end{tikzpicture}
  \caption{Sprungantwort eines PT1-Glieds: $G(p)=\frac{K}{1+Tp}$, $x_a(t)=K(1-e^{-t/T})\sigma(t)$.}
  \label{fig:pt1_step}
\end{figure}


\paragraph{Beispiel (PT2 aus DGL).}
\[
  T^2\,\ddot x_a + 2DT\,\dot x_a + x_a = K\,x_e
  \quad\Longrightarrow\quad
  (T^2p^2+2DTp+1)X_a(p)=K\,X_e(p),
\]
\[
  G(p)=\frac{K}{1+2DTp+T^2p^2}.
\]

Weitere, umfangreichere \"Ubungsaufgaben finden sich im Anhang, siehe Abschnitt~\ref{app:aufgaben}.

\subsection{\"Ubersicht: DGL zu \"UTF}
\begin{center}
  \begin{tabular}{@{}lll@{}}
    \toprule
    Name & DGL & \"Ubertragungsfunktion $G(p)$ \\
    \midrule
    P-Glied   & $x_a = K_P\,x_e$ & $K_P$ \\
    D-Glied   & $x_a = K_D\,\dot x_e$ & $K_D\,p$ \\
    I-Glied   & $\dot x_a = K_I\,x_e$ & $\displaystyle\frac{K_I}{p}$ \\
    DT1-Glied & $T\,\dot x_a + x_a = K_D\,\dot x_e$ & $\displaystyle \frac{K_D\,p}{1+Tp}$ \\
    PT1-Glied & $T\,\dot x_a + x_a = K\,x_e$ & $\displaystyle \frac{K}{1+Tp}$ \\
    PT2-Glied & $T^2\,\ddot x_a + 2DT\,\dot x_a + x_a = K\,x_e$ & $\displaystyle \frac{K}{1+2DTp+T^2p^2}$ \\
    \bottomrule
  \end{tabular}
\end{center}
