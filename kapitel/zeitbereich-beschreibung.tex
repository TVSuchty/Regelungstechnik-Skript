% kapitel/zeitbereich-beschreibung.tex
\section{Mathematische Beschreibung im Zeitbereich}
\subsection{Lineare Differentialgleichung mit konstanten Koeffizienten}
Ein LZI-\"UTG kann im Zeitbereich h\"aufig durch eine lineare Differentialgleichung (DGL)
beschrieben werden:
\[
  \sum_{m=0}^{n} a_m\,\frac{\dd^{m}x_a(t)}{\dd t^{m}}
  =
  \sum_{k=0}^{\ell} b_k\,\frac{\dd^{k}x_e(t)}{\dd t^{k}},
  \qquad a_n\neq 0.
\]

\paragraph{Beispiel (PT1).}
Ein Verz\"ogerungsglied erster Ordnung erf\"ullt
$T\,\dot x_a + x_a = K\,x_e$.
F\"ur einen Einheitssprung $x_e(t)=\sigma(t)$ ergibt sich eine exponentielle Ann\"aherung
an den Endwert $K$.

\subsection{Beispiele: typische \"UTG}
\begin{center}
  \begin{tabular}{@{}lll@{}}
    \toprule
    Name & Zeitbereich (DGL) & Bemerkung \\
    \midrule
    P-Glied   & $x_a = K_P\,x_e$ & proportional \\
    I-Glied   & $\dot x_a = K_I\,x_e$ & integratorisch \\
    D-Glied   & $x_a = K_D\,\dot x_e$ & ideal differenzierend \\
    PT1-Glied & $T\,\dot x_a + x_a = K\,x_e$ & Verz\"ogerung 1.\ Ordnung \\
    PT2-Glied & $T^2\,\ddot x_a + 2D T\,\dot x_a + x_a = K\,x_e$ & Verz\"ogerung 2.\ Ordnung \\
    DT1-Glied & $T\,\dot x_a + x_a = K_D\,\dot x_e$ & realisierbares D-Glied \\
    \bottomrule
  \end{tabular}
\end{center}
