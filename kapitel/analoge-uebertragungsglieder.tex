% kapitel/analoge-uebertragungsglieder.tex
\section{Analoge \"Ubertragungsglieder (\"UTG)}
\subsection{Begriffe und Eigenschaften}
Ein \"Ubertragungsglied (\"UTG) beschreibt den Zusammenhang zwischen einem Eingangssignal
(\glqq Ursache\grqq) $x_e(t)$ und einem Ausgangssignal (\glqq Wirkung\grqq) $x_a(t)$:
\[
  x_e(t)\;\longrightarrow\;\text{\"UTG}\;\longrightarrow\;x_a(t).
\]
Im Folgenden betrachten wir in erster Linie analoge, \textbf{lineare} und \textbf{zeitinvariante} Systeme
(\textbf{LZI}-Systeme). F\"ur diese Systeme gelten besonders einfache Rechenregeln
(Superposition, Faltung, Laplace- und Frequenzbereichsdarstellung).

\paragraph{Linearit\"at (Superpositionsprinzip).}
Ein System ist linear, wenn f\"ur beliebige Signale $x_1, x_2$ und Konstanten $a,b$ gilt:
\[
  x_e(t)=a\,x_1(t)+b\,x_2(t)\quad\Rightarrow\quad
  x_a(t)=a\,y_1(t)+b\,y_2(t),
\]
wobei $y_i(t)$ die jeweilige Antwort auf $x_i(t)$ ist.

\paragraph{Zeitinvarianz.}
Ein System ist zeitinvariant, wenn eine zeitliche Verschiebung des Eingangssignals
eine identische Verschiebung des Ausgangssignals bewirkt (bei gleichen Anfangsbedingungen):
\[
  x_e^{\ast}(t)=x_e(t-t_0)\quad\Rightarrow\quad x_a^{\ast}(t)=x_a(t-t_0).
\]

\paragraph{Beispiel (LZI).}
Ein Verst\"arker mit $x_a(t)=2\,x_e(t)$ ist linear und zeitinvariant.
Eine Rampe $x_e(t)=t\,\sigma(t)$ wird zu $x_a(t)=2t\,\sigma(t)$.
