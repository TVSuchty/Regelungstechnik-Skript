% kapitel/reglerstrukturen-realisierung.tex
\section{Reglerstrukturen (PD, PID) und Realisierungshinweise}
\subsection{PD- und PID-Formen}
\paragraph{PD-Regler.}
\[
  G_R(p)=K_{PR}+K_{DR}p
  =K_{PR}\left(1+T_V p\right),
  \qquad
  T_V=\frac{K_{DR}}{K_{PR}}.
\]

\paragraph{Realisierbarer PD (mit Realisierungspol).}
Ein idealer Differenzierer verst\"arkt hohe Frequenzen unbegrenzt; daher wird h\"aufig ein Pol eingef\"uhrt, z.\,B.
\[
  G_R(p)=K_{PR}\left(1+\frac{T_V p}{1+T_R p}\right),
\]
wobei $T_R$ die Zeitkonstante des Realisierungspols ist. Der Realisierungspol wirkt als Tiefpass
im D-Anteil: f\"ur kleine Frequenzen gilt nahezu $T_V p$, f\"ur hohe Frequenzen wird die
Verst\"arkung begrenzt (Rauschunterdr\"uckung, physikalische Realisierbarkeit).

\paragraph{PID-Regler (ideal).}
\[
  G_R(p)=K_{PR}+\frac{K_{IR}}{p}+K_{DR}p
  =K_{PR}\left(1+\frac{1}{T_N p}+T_V p\right),
  \qquad
  T_N=\frac{K_{PR}}{K_{IR}}.
\]

\paragraph{PID-Regler (realisierbar).}
\[
  G_R(p)=K_{PR}\left(1+\frac{1}{T_N p}+\frac{T_V p}{1+T_R p}\right),
  \qquad T_R \ll T_V.
\]
Hier ist $T_N$ die Nachstellzeit (I-Anteil) und $T_R$ der Realisierungspol des D-Anteils.

\subsection{(Invertierender) OPV-Grundsatz}
F\"ur einen idealen invertierenden Operationsverst\"arker gilt im Laplace-Bereich:
\[
  \frac{U_a(p)}{U_e(p)}=-\frac{Z_2(p)}{Z_1(p)}.
\]
Durch geeignete Wahl von $Z_1,Z_2$ (Widerst\"ande/Kondensatoren) lassen sich P-, I-, D- und DT1-Verhalten realisieren.

\begin{center}
  \begin{tabular}{@{}llll@{}}
    \toprule
    Funktion & $Z_1(p)$ & $Z_2(p)$ & $G(p)=-Z_2/Z_1$ \\
    \midrule
    P   & $R_1$ & $R_0$ & $-\dfrac{R_0}{R_1}$ \\
    I   & $R_1$ & $\dfrac{1}{C_0p}$ & $-\dfrac{1}{R_1C_0p}$ \\
    D   & $\dfrac{1}{C_1p}$ & $R_0$ & $-R_0C_1p$ \\
    DT1 & $R_1+\dfrac{1}{C_1p}$ & $R_0$ & $-\dfrac{R_0C_1p}{1+R_1C_1p}$ \\
    \bottomrule
  \end{tabular}
\end{center}

\paragraph{Beispiel (I-Glied).}
Mit $R_1=10\,\text{k}\Omega$ und $C_0=1\,\mu\text{F}$ ergibt sich
$G(p)=-\dfrac{1}{R_1C_0p}=-\dfrac{1}{0{,}01\,p}$.
