% -----------------------------------------------------------------------------
% Zusammenfassung handschriftlicher Notizen: Entwurf analoger Regler
% (als Einfuege-Abschnitt fuer das Skript)
% -----------------------------------------------------------------------------

\subsection{Entwurf analoger Regler}

\subsubsection{Guetekriterien}
Zur Bewertung des Zeitverhaltens (Fuehrungs- und Stoerantwort) werden oft
Integralguetekriterien ueber den Regelfehler $e(t)$ verwendet:
\begin{align}
  IAE  &= \int_{0}^{\infty} \lvert e(t)\rvert\,\mathrm{d}t,\\
  ISE  &= \int_{0}^{\infty} e^{2}(t)\,\mathrm{d}t,\\
  ITAE &= \int_{0}^{\infty} t\,\lvert e(t)\rvert\,\mathrm{d}t.
\end{align}

\subsubsection{Reglerentwurf fuer gewuenschtes Folgeverhalten}
Der betrachtete einschleifige Regelkreis (Einheitsrueckfuehrung) lautet
schematisch
\[\quad w \rightarrow \boxed{G_R(p)} \rightarrow \boxed{G_S(p)} \rightarrow x,\qquad e=w-x.\]

\paragraph{Vorgehen (prinzipiell).}
\begin{enumerate}
  \item Streckenuebertragungsfunktion $G_S(p)$ ermitteln.
  \item Gewuenschte Parameter der geschlossenen Fuehrungsuebertragungsfunktion
        $G_w(p)=x(p)/w(p)$ festlegen (insbesondere Daempfung und Zeitmass).
  \item Reglerparameter berechnen und anschliessend die Kennwerte des
        geschlossenen Regelkreises pruefen.
\end{enumerate}

\paragraph{Gewuenschte Fuehrungs-UTF (Ansatz).}
In den Notizen wird ein PT2-Verhalten des geschlossenen Kreises angesetzt:
\begin{equation}
  G_{w,\mathrm{soll}}(p)=\frac{1}{1+2D_{gw}T_{gw}p+T_{gw}^{2}p^{2}}.
\end{equation}
Wird zusaetzlich die statische Kreisverstaerkung $V_0=G_0(0)$ (mit
$G_0(p)=G_R(p)G_S(p)$) beruecksichtigt, wird $G_w(p)$ naeherungsweise
in der Form
\begin{equation}
  G_w(p)\approx \frac{V_0}{1+V_0}\,\frac{1}{1+2D_{gw}T_{gw}p+T_{gw}^{2}p^{2}}
\end{equation}
verwendet.
Damit gilt: Bei endlichem $V_0$ ist der statische Verstaerkungsfaktor
$V_0/(1+V_0)<1$, waehrend fuer PI- bzw. PID-Regler aufgrund des
Integratoranteils typischerweise $V_0\to\infty$ und damit
$\lim_{p\to 0}G_w(p)=1$ resultiert.

\subsubsection{Hinweis zur Anwendbarkeit}
Das in den Notizen skizzierte Verfahren ist \emph{nicht} uneingeschraenkt
auf alle Strecken (insbesondere stark schwingfaehige oder unsichere Modelle)
uebertragbar.

\subsubsection{Strecken in Entwurfsform bringen (PT3$^{\ast}$-Form)}
Fuer die anschliessende Parameterberechnung wird die Strecke naeherungsweise
in eine PT3$^{\ast}$-Form ueberfuehrt:
\begin{equation}
  G_S(p)\approx \frac{K_S}{(1+T_1p)(1+T_2p)(1+T_3^{\ast}p)}.
\end{equation}
Diese Form kann naeherungsweise aus unterschiedlichen Strecken-UTF gewonnen
werden, indem Summanden zusammengefasst und ggf. Verhaeltnisglieder
(Nullstellen) beruecksichtigt werden, z.\,B.
\begin{align}
  &G_S(p)=\frac{K_S}{1+2D_ST_Sp+T_S^{2}p^{2}}, &&(D_S<1)\\
  &G_S(p)=K_S\,\frac{1+T_Dp}{(1+T_1p)(1+T_2p)}\,e^{-T_t p},\\
  &G_S(p)=K_S\,\frac{\prod_k(1+T_{Dk}p)}{\prod_i(1+T_ip)}\,e^{-T_t p},\\
  &G_S(p)=\frac{K_S}{(1+Tp)^n},\qquad n\ge 3.
\end{align}

\paragraph{Effektive Zeitkonstante $T_3^{\ast}$.}
In den Notizen wird $T_3^{\ast}$ durch Zusammenfassen bzw. Umformung
(inkl. Totzeit/Transportzeit) gebildet. Als Merkschema:
\begin{equation}
  T_3^{\ast} \approx (T_3+T_4+\dots)\; -\; (T_{D1}+T_{D2}+\dots)\; +\; T_t.
\end{equation}

\paragraph{Integrierende Strecke (Sonderfall).}
Fuer integrierende Strecken tritt als Ausgangsform z.\,B.
$G_S(p)=K_{iS}/p\,(\dots)$ auf; hier kann ein P-Regler (anstatt PI)
bereits eine endliche Regelabweichung vermeiden.

\subsubsection{Sonderfall: PT2 mit definierter Daempfung}
Wird eine (ueberdaempfte) PT2-Strecke betrachtet,
\begin{equation}
  G_S(p)=K_S\,\frac{1}{1+2D_ST_Sp+T_S^{2}p^{2}},\qquad D_S>1,
\end{equation}
so laesst sich der Nenner faktorisieren:
\begin{equation}
  1+2D_ST_Sp+T_S^{2}p^{2}=(1+T_1p)(1+T_2p)
  =1+(T_1+T_2)p+T_1T_2p^{2}.
\end{equation}
In diesem Fall wird in den Notizen $T_3^{\ast}=0$ gesetzt.
\medskip

\noindent\textbf{Beachte:} PD- und PID-Regler sind nach dem in den Notizen
verwendeten Berechnungsalgorithmus fuer diesen Spezialfall nicht geeignet,
weil dabei $K_{PR}\to\infty$ resultieren kann.

\subsubsection{Reglerstrukturen und Parameter (aus der Notiz-Tabelle)}
Es werden die Standardstrukturen
\begin{align}
  \text{P:}\quad &G_R(p)=K_{PR},\\
  \text{PI:}\quad &G_R(p)=K_{PR}\Bigl(1+\frac{1}{T_N p}\Bigr)=K_{PR}+\frac{K_{IR}}{p},\\
  \text{PD:}\quad &G_R(p)=K_{PR}(1+T_V p)=K_{PR}+K_{DR}p,\\
  \text{PID:}\quad &G_R(p)=K_{PR}\Bigl(1+\frac{1}{T_N p}+T_V p\Bigr)=K_{PR}+\frac{K_{IR}}{p}+K_{DR}p
\end{align}
verwendet. Fuer die PT3$^{\ast}$-Streckenform (mit $T_1,\,T_2,\,T_3^{\ast}$)
geben die Notizen folgende Parameterzusammenhaenge an (mit $T_h$ als
Zeitmass des gewuenschten Folgeverhaltens; in vielen Skripten entspricht dies
$T_h\equiv T_{gw}$):

\begin{center}
\renewcommand{\arraystretch}{1.4}
\begin{tabular}{lccc}
\hline
 & PI & PD & PID\\
\hline
$K_{PR}$
  & $\displaystyle \frac{T_h}{K_S\,(T_2+T_3^{\ast})\,4D_{gw}^{2}}$
  & $\displaystyle \frac{T_h}{K_S\,T_3^{\ast}\,4D_{gw}^{2}}$
  & $\displaystyle \frac{T_h+T_2}{K_S\,T_3^{\ast}\,4D_{gw}}$\\
$T_N$ & $T_N=T_h$ & -- & $T_N=T_1+T_2$\\
$T_V$ & -- & $T_V=T_2$ & $\displaystyle T_V=\frac{T_1T_2}{T_1+T_2}$\\
\hline
\end{tabular}
\end{center}

Die Umrechnung auf Parallelform-Parameter erfolgt ueber
\begin{equation}
  K_{IR}=\frac{K_{PR}}{T_N},\qquad K_{DR}=K_{PR}\,T_V.
\end{equation}

\subsubsection{Beispiel- und Uebungsangaben (aus den Notizen)}
\paragraph{Beispiel (mit Totzeit).}
\begin{equation}
  G_S(p)=10\,\frac{1+0.2p}{(1+p)(1+0.5p)(1+0.4p)}\,e^{-0.5p}.
\end{equation}
Gesucht sind P-, PI-, PD- und PID-Regler. In den Notizen wird dazu eine
Tabelle zur Gegenueberstellung der Reglerparameter (u.\,a. $K_{PR}$, $T_N$,
$T_V$, $K_{IR}$, $K_{DR}$ sowie pruefbare Kennwerte des geschlossenen Kreises)
angelegt.

\paragraph{Ausschnitt aus einer Beispielrechnung (PD).}
Fuer einen PD-Regler wurden notiert:
\(K_{PR}=0.083\), \(T_V=10\,\mathrm{s}\) und damit \(K_{DR}=0.83\).
Weiterhin: \(G_w(0)=1\) und \(T_{RK}\approx 1.5\,\mathrm{s}\).

\paragraph{Hausaufgabe (integrierende Strecke).}
\begin{equation}
  G_S(p)=\frac{3}{p}\,\frac{1+p}{(1+4p)(1+7p)(1+6.5p)}.
\end{equation}
(Angabe in den Notizen: Entwurf eines P- bzw. PD-Reglers fuer gewuenschte
Kenngroessen des Folgeverhaltens.)

