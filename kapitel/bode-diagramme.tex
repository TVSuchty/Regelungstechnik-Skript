% kapitel/bode-diagramme.tex
\section{Bode-Diagramme}
\subsection{Amplitude in Dezibel}
\[
  L(\omega)=20\log_{10}\!\bigl(\lvert G(\jj\omega)\rvert\bigr)\ \si{dB}.
\]
Wichtiger Vorteil:
\[
  G(\jj\omega)=G_1(\jj\omega)\,G_2(\jj\omega)
  \quad\Rightarrow\quad
  L(\omega)=L_1(\omega)+L_2(\omega).
\]
Analog addieren sich Phasen:
\[
  \varphi(\omega)=\varphi_1(\omega)+\varphi_2(\omega).
\]

\subsection{Beispiel: PT1-Glied}
\[
  G(\jj\omega)=\frac{K}{1+\jj\omega T}.
\]
Damit:
\[
  \lvert G(\jj\omega)\rvert=\frac{K}{\sqrt{1+(\omega T)^2}},
  \qquad
  \varphi(\omega)=-\arctan(\omega T).
\]
Asymptotisch:
\begin{itemize}[itemsep=2pt]
  \item $\omega\ll 1/T$: $\lvert G\rvert\approx K$ (0\,dB-Steigung), $\varphi\approx 0^\circ$.
  \item $\omega\gg 1/T$: $\lvert G\rvert\approx \frac{K}{\omega T}$ (Steigung \SI{-20}{dB/dec}), $\varphi\approx -90^\circ$.
\end{itemize}


% Bode diagram of a PT1 element
\begin{figure}[tb]
  \centering
  \begin{tikzpicture}
    \begin{groupplot}[
      group style={group size=1 by 2, vertical sep=0.9cm},
      width=0.78\textwidth, height=0.34\textwidth,
      grid=both, grid style={line width=.1pt, draw=gray!20},
      major grid style={line width=.2pt, draw=gray!35},
      xmode=log,
      xmin=1e-2, xmax=1e2,
      axis lines=left,
      tick label style={font=\small},
      label style={font=\small},
    ]
      \nextgroupplot[ylabel={$L(\omega)$ / dB}, title={Bode-Diagramm (PT1, $K=1,\,T=1$)}]
        \addplot[thick] table[x=w,y=mag_db,col sep=comma] {data/pt1_bode_K1_T1.csv};
        \addplot[dashed] coordinates {(1,0) (1e2,-40)};
        \addplot[dashed] coordinates {(1e-2,0) (1,0)};
      \nextgroupplot[xlabel={$\omega T$}, ylabel={$\varphi(\omega)$ / $^\circ$}, ymin=-100, ymax=10]
        \addplot[thick] table[x=w,y=phase_deg,col sep=comma] {data/pt1_bode_K1_T1.csv};
        \addplot[dashed] coordinates {(1,-45) (1,-45)};
        \addplot[dashed] coordinates {(1,-90) (1e2,-90)};
    \end{groupplot}
  \end{tikzpicture}
  \caption{Bode-Diagramm des PT1-Glieds. Die gestrichelten Linien zeigen die asymptotischen Näherungen (0~dB bzw.\ $-20$~dB/dec, Phase gegen $-90^\circ$).}
  \label{fig:pt1_bode}
\end{figure}


\paragraph{Beispiel (Verst\"arkung).}
Wird $K$ von $1$ auf $10$ erh\"oht, verschiebt sich der Amplitudengang um $20\,\mathrm{dB}$
nach oben; die Phasenlage bleibt unver\"andert.

\FloatBarrier

