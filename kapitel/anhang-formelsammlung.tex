% kapitel/anhang-formelsammlung.tex
\section{Mini-Formelsammlung (Laplace)}
\begin{center}
  \begin{tabular}{@{}ll@{}}
    \toprule
    Zeitfunktion $x(t)$ (f\"ur $t\ge 0$) & $X(p)=\mathcal{L}\{x(t)\}$ \\
    \midrule
    $\sigma(t)$ & $\dfrac{1}{p}$ \\
    $\delta(t)$ & $1$ \\
    $t\,\sigma(t)$ & $\dfrac{1}{p^2}$ \\
    $\mathrm{e}^{-at}$ $(a>0)$ & $\dfrac{1}{p+a}$ \\
    $\sin(\omega t)$ & $\dfrac{\omega}{p^2+\omega^2}$ \\
    $\cos(\omega t)$ & $\dfrac{p}{p^2+\omega^2}$ \\
    \bottomrule
  \end{tabular}
\end{center}

\section{Ausf\"uhrliche Aufgaben}
\label{app:aufgaben}
\begin{enumerate}[itemsep=4pt]
  \item \textbf{\"Ubertragungsfunktion.}
  Gegeben ist
  $T\dot x_a + x_a = K\,x_e + K_D\,\dot x_e$ mit Null-Anfangsbedingungen.
  Bestimmen Sie $G(p)$ und die Sprungantwort f\"ur $x_e(t)=\sigma(t)$.
  \item \textbf{Bode-Absch\"atzung.}
  F\"ur $G(p)=\dfrac{10}{(1+0{,}5p)(1+2p)}$ skizzieren Sie die asymptotischen
  Betragsgeraden und geben Sie die Eckfrequenzen an.
  \item \textbf{Regelkreis.}
  F\"ur $G_R(p)=K_{PR}\left(1+\dfrac{1}{T_N p}\right)$ und
  $G_S(p)=\dfrac{K_S}{1+Tp}$ bestimmen Sie $G_0(p)$, $G_w(p)$ und den
  statischen Regelfaktor $R$.
\end{enumerate}
