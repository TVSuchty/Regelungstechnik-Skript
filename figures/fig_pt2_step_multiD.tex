% Step response of PT2 for different damping ratios D
\begin{figure}[tb]
  \centering
  \begin{tikzpicture}
    \begin{axis}[
      width=0.78\textwidth, height=0.45\textwidth,
      grid=both, grid style={line width=.1pt, draw=gray!20},
      major grid style={line width=.2pt, draw=gray!35},
      xlabel={$t/T$}, ylabel={$x_a(t)/K$},
      xmin=0, xmax=20,
      ymin=-0.05, ymax=1.8,
      axis lines=left,
      legend style={font=\small, at={(0.98,0.02)}, anchor=south east},
      tick label style={font=\small},
      label style={font=\small},
    ]
      \addplot[thick] table[x=t,y=D0.2,col sep=comma] {data/pt2_step_T1_K1_multiD.csv};
      \addlegendentry{$D=0{,}2$}
      \addplot[thick, densely dashed] table[x=t,y=D0.5,col sep=comma] {data/pt2_step_T1_K1_multiD.csv};
      \addlegendentry{$D=0{,}5$}
      \addplot[thick, dashdotted] table[x=t,y=D0.8,col sep=comma] {data/pt2_step_T1_K1_multiD.csv};
      \addlegendentry{$D=0{,}8$}
      \addplot[thick, dotted] table[x=t,y=D1.0,col sep=comma] {data/pt2_step_T1_K1_multiD.csv};
      \addlegendentry{$D=1{,}0$}
      \addplot[thick] table[x=t,y=D1.2,col sep=comma] {data/pt2_step_T1_K1_multiD.csv};
      \addlegendentry{$D=1{,}2$}
      \addplot[dashed] coordinates {(0,1) (20,1)};
    \end{axis}
  \end{tikzpicture}
  \caption{Sprungantwort eines normierten PT2-Glieds ($T=1$, $K=1$) f\"ur unterschiedliche D\"ampfungsma\ss e $D$. Unterd\"ampfung ($D<1$) f\"uhrt zu \"Uberschwingen und Schwingungen; bei $D\ge1$ verschwindet das \"Uberschwingen.}
  \label{fig:pt2_step_multiD}
\end{figure}
