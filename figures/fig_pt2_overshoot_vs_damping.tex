
% Overshoot vs damping ratio
\begin{figure}[tb]
  \centering
  \begin{tikzpicture}
    \begin{axis}[
      width=0.70\textwidth, height=0.42\textwidth,
      grid=both, grid style={line width=.1pt, draw=gray!20},
      major grid style={line width=.2pt, draw=gray!35},
      xlabel={Dämpfungsmaß $D$}, ylabel={$M_p$ / \%},
      xmin=0, xmax=1,
      ymin=0, ymax=110,
      axis lines=left,
      tick label style={font=\small},
      label style={font=\small},
    ]
      \addplot[thick] table[x=D,y=Mp_percent,col sep=comma] {data/pt2_overshoot_vs_damping.csv};
      \addplot[dashed] coordinates {(0.5,0) (0.5,110)};
      \node[anchor=west, font=\small] at (axis cs:0.51,70) {$D=0{,}5 \Rightarrow M_p\approx16\%$};
    \end{axis}
  \end{tikzpicture}
  \caption{Relatives Überschwingen $M_p=\exp\!\left(-\frac{D\pi}{\sqrt{1-D^2}}\right)$ (unterdämpfter Fall $0<D<1$). Mit wachsender Dämpfung sinkt das Überschwingen stark.}
  \label{fig:overshoot_vs_damping}
\end{figure}
