
% Magnitude response of PT2 for different damping ratios D
\begin{figure}[tb]
  \centering
  \begin{tikzpicture}
    \begin{axis}[
      width=0.78\textwidth, height=0.45\textwidth,
      grid=both, grid style={line width=.1pt, draw=gray!20},
      major grid style={line width=.2pt, draw=gray!35},
      xmode=log,
      xlabel={$\omega T$}, ylabel={$L(\omega)$ / dB},
      xmin=1e-2, xmax=1e2,
      ymin=-60, ymax=20,
      axis lines=left,
      legend style={font=\small, at={(0.02,0.02)}, anchor=south west},
      tick label style={font=\small},
      label style={font=\small},
    ]
      \addplot[thick] table[x=w,y=D0.2,col sep=comma] {data/pt2_mag_bode_T1_K1_multiD.csv};
      \addlegendentry{$D=0{,}2$}
      \addplot[thick, densely dashed] table[x=w,y=D0.5,col sep=comma] {data/pt2_mag_bode_T1_K1_multiD.csv};
      \addlegendentry{$D=0{,}5$}
      \addplot[thick, dashdotted] table[x=w,y=D0.8,col sep=comma] {data/pt2_mag_bode_T1_K1_multiD.csv};
      \addlegendentry{$D=0{,}8$}
      \addplot[thick, dotted] table[x=w,y=D1.2,col sep=comma] {data/pt2_mag_bode_T1_K1_multiD.csv};
      \addlegendentry{$D=1{,}2$}
      \addplot[dashed] coordinates {(1,0) (1,0)};
    \end{axis}
  \end{tikzpicture}
  \caption{Betragsfrequenzgang eines normierten PT2-Glieds ($T=1$, $K=1$). Kleine Dämpfung erzeugt eine Resonanzüberhöhung (Peak im Betrag), die im Zeitbereich typischerweise mit starkem Überschwingen korreliert.}
  \label{fig:pt2_mag_multiD}
\end{figure}
