% Nyquist curves of PT elements (zoom around origin)
\begin{figure}[H]
  \centering
  \begin{tikzpicture}
    \begin{axis}[
      width=0.82\linewidth,
      height=0.55\linewidth,
      axis lines=middle,
      axis equal image,
      xmin=-0.3, xmax=0.3,
      ymin=-0.3, ymax=0.3,
      grid=both,
      grid style={line width=.1pt, draw=gray!20},
      major grid style={line width=.2pt, draw=gray!35},
      tick label style={font=\small},
      label style={font=\small},
      xlabel={$\Ree$},
      ylabel={$\Imm$},
      legend style={font=\small, at={(0.02,0.98)}, anchor=north west},
    ]
      \addplot[thick, blue] table[x=re,y=im,col sep=comma] {data/nyquist_PT1_K1_T1.csv};
      \addlegendentry{PT1}
      \addplot[thick, red] table[x=re,y=im,col sep=comma] {data/nyquist_PT2_K1_T1.csv};
      \addlegendentry{PT2}
      \addplot[thick, green!60!black] table[x=re,y=im,col sep=comma] {data/nyquist_PT3_K1_T1.csv};
      \addlegendentry{PT3}
      \addplot[thick, orange] table[x=re,y=im,col sep=comma] {data/nyquist_PT4_K1_T1.csv};
      \addlegendentry{PT4}
      \addplot[thick, teal!70!black] table[x=re,y=im,col sep=comma] {data/nyquist_PT7_K1_T1.csv};
      \addlegendentry{PT7}
      \filldraw (0,0) circle (1.2pt);
    \end{axis}
  \end{tikzpicture}
  \caption{Zoom um den Ursprung: Nyquist-Ortskurven der PT-Glieder ($K=1$, $T=1$; PT1--PT4 und PT7).
  Mit wachsendem $n$ verl\"auft die Kurve n\"aher um $(0,0)$ und durchl\"auft dabei mehrere
  Quadranten (nur $\omega>0$, Spiegelung an der reellen Achse ergibt die vollst\"andige Kurve).}
  \label{fig:nyquist_standardglieder_zoom0}
\end{figure}
