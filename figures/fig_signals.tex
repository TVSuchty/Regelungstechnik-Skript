
% Typical input signals (step, ramp, rectangular pulse)
\begin{figure}[tb]
  \centering
  \begin{tikzpicture}
    \begin{groupplot}[
      group style={group size=2 by 2, horizontal sep=1.4cm, vertical sep=1.0cm},
      width=0.46\textwidth, height=0.28\textwidth,
      grid=both, grid style={line width=.1pt, draw=gray!20},
      major grid style={line width=.2pt, draw=gray!35},
      xlabel={$t$}, xmin=-0.5, xmax=2,
      ymin=-0.1, ymax=2.1,
      axis lines=left,
      tick label style={font=\small},
      label style={font=\small},
      title style={font=\small},
    ]
      \nextgroupplot[title={Sprung $\sigma(t)$}, ylabel={$x_e(t)$}]
        \addplot[thick] table[x=t,y=step,col sep=comma] {data/signals_step_ramp_pulse.csv};
        \addplot[only marks, mark=*, mark size=1.2pt] coordinates {(0,0)}; % discontinuity marker
      \nextgroupplot[title={Rampe $r(t)=t\,\sigma(t)$}]
        \addplot[thick] table[x=t,y=ramp,col sep=comma] {data/signals_step_ramp_pulse.csv};
      \nextgroupplot[title={Rechteckimpuls $\sigma(t)-\sigma(t-t_0)$}]
        \addplot[thick] table[x=t,y=pulse,col sep=comma] {data/signals_step_ramp_pulse.csv};
        \addplot[dashed] coordinates {(1,0) (1,1)};
      \nextgroupplot[title={Diracimpuls $\delta(t)$}]
        \addplot[thick, ->] coordinates {(0,0) (0,1.6)};
        \addplot[only marks, mark=*, mark size=1.2pt] coordinates {(0,0)};
        \node[anchor=west, font=\small] at (axis cs:0.05,1.55) {$\delta(t)$};
    \end{groupplot}
  \end{tikzpicture}
  \caption{Typische Eingangssignale im Zeitbereich (normiert, $t_0=1$) inklusive Diracimpuls.}
  \label{fig:signals_typical}
\end{figure}






