% Nyquist curves of standard elements (K=1, T=1)
\begin{figure}[H]
  \centering
  \begin{tikzpicture}
    \begin{axis}[
      width=0.9\linewidth,
      height=0.6\linewidth,
      axis lines=middle,
      axis equal image,
      xmin=-0.6, xmax=1.1,
      ymin=-1.2, ymax=1.2,
      grid=both,
      grid style={line width=.1pt, draw=gray!20},
      major grid style={line width=.2pt, draw=gray!35},
      tick label style={font=\small},
      label style={font=\small},
      xlabel={$\Ree$},
      ylabel={$\Imm$},
      legend style={font=\small, at={(0.0,1.02)}, anchor=south west, legend columns=3},
    ]
      \addplot[thick, blue] table[x=re,y=im,col sep=comma] {data/nyquist_PT1_K1_T1.csv};
      \addlegendentry{PT1}
      \addplot[thick, red] table[x=re,y=im,col sep=comma] {data/nyquist_PT2_K1_T1.csv};
      \addlegendentry{PT2}
      \addplot[thick, green!60!black] table[x=re,y=im,col sep=comma] {data/nyquist_PT3_K1_T1.csv};
      \addlegendentry{PT3}
      \addplot[thick, orange] table[x=re,y=im,col sep=comma] {data/nyquist_PT4_K1_T1.csv};
      \addlegendentry{PT4}
      \addplot[thick, teal!70!black] table[x=re,y=im,col sep=comma] {data/nyquist_PT7_K1_T1.csv};
      \addlegendentry{PT7}
      \addplot[thick, gray, dashed, restrict y to domain=-1.2:1.2] table[x=re,y=im,col sep=comma] {data/nyquist_I_K1.csv};
      \addlegendentry{I}
      \addplot[thick, black, dotted, restrict y to domain=-1.2:1.2] table[x=re,y=im,col sep=comma] {data/nyquist_D_K1.csv};
      \addlegendentry{D}
      \addplot[thick, cyan!60!black, dashdotdotted, restrict y to domain=-1.2:1.2] table[x=re,y=im,col sep=comma] {data/nyquist_PI_K1_T1.csv};
      \addlegendentry{PI}
      \addplot[thick, brown, dashed, restrict y to domain=-1.2:1.2] table[x=re,y=im,col sep=comma] {data/nyquist_PD_K1_T1.csv};
      \addlegendentry{PD}
      \addplot[only marks, mark=*, mark size=1.6pt] coordinates {(1,0)};
      \addlegendentry{P}
    \end{axis}
  \end{tikzpicture}
  \caption{Nyquist-Ortskurven der Standardglieder (Ausschnitt, $K=1$, $T=1$; PT1--PT4, PT7,
  PI und PD). Gezeigt ist
  $\omega>0$; die vollst\"andige Kurve ergibt sich durch Spiegelung an der reellen Achse.
  Das I- und D-Glied verlaufen entlang der imagin\"aren Achse.}
  \label{fig:nyquist_standardglieder}
\end{figure}
